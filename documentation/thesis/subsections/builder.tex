\subsection{Builder}

Amennyiben van egy komplex objektumunk, ami több komponenst is magában foglalhat, érdemes elválasztani az objektum létrehozását az objektum viselkedésének implementációjától \cite{gamma1995elements} . Amennyiben az objektum módot ad 

\begin{listing}
	\item egy egyszerú alapreprezentáció létrehozására
	\item ezen reprezentáció kibővítésére
\end{listing}

létre tudunk hozni egy entitást, ami felügyeli ezen folyamatot. Ez az entitás, a \emph{Builder}, definiál egy létrehozási folyamatot, mely folyamat több, eltérő reprezentációt is létre tud hozni, és az ezt egységesítő interfészt. \par

Egy példa a Java standard könyvtárából a \code{StringBuilder}. Segítségével képesek vagyunk egy stringet felépíteni kisebb szerkezeti egységek hozzáadásával, majd az így kapott karaktersorozatot megkapni. Az \code{insert} és \code{append} metódusai számos overloaddal rendelkeznek, így rugalmasabban tudjuk felépíteni a kívánt végeredményt. \par

Egy további példa a projectből a \code{NavBar} és a \code{NavBarBuilder}. A \code{NavBar} egy egységes navigációs sáv komponens, amelyet minden felhasználói csoport felülete használ. Egy opcionális címkéből, ami valamilyen információt ad a felületről, ahol megjelenik a sáv, illetve kattintható gombok sorozatából áll, melyek az alkalmazás egy megfelelő felületére navigálják a felhasználót. A \code{NavBar} egy komplex objektum. Ez nem viselkedésében tükröződik, hanem abban, hogy a megadható navigációs opciók számát nem tudjuk egyértelműen deifniálni, igény szerint változhat.\par

Egy lehetséges megoldás lenne, ha több konstruktort hoznánk létre az osztálynak, mindegyikben kezelve egy-egy lehetséges igényt: \par

\begin{listing}
	\item ne legyen címkéje, ne legyenek opciók
	\item legyen címkéje, de ne legyenek opciók
	\item ne legyen címkéje, legyen valamennyi opciója
	\item legyen címkéje, legyen valamennyi opciója
\end{listing} \par

Ezen esetek számát le tudjuk csökkenteni, ha a konstruktor \par

\begin{listing}
	\item a címkét \code{Optional<String>} típusúként kezeli, azaz függetlenül attól, hogy kell-e címke vagy sem, a konstruktor rendelkezni fog ezzel a paraméterrel
	\item az opciókat változó hosszúságú paraméterlistaként adjuk meg, amely, abban az esetben, ha nincs navigációs opció, üres, másképp pedig az opciókat tartalmazza.
\end{listing} \par

\snippet{NavBar}{A NavBar osztály implementációja}

Ezzel lecsökkentettük az objektum létrehozásának módjainak számát, azonban ez a megoldás nem rugalmas. Az objektum létrehozásához minden információra szükségünk van ami a címkét és az opciókat illeti, és lehetséges, hogy ezek nem állnak rendelkezésre egyszerre, vagy csak egy részük és ezek áthídalása jelentősen növeli a kódbázisunk komplexitását. \par

A \code{NavBarBuilder} bevezetésével a létrehozási folyamat jelentősen leegyszerűsödik. Definiálunk egy egységes interfészt, amely tartalmazza a \code{NavBar} létrehozásának lépéseit.

\snippet{NavBarBuilder}{NavBarBuilder}

Ezek a metódusok nem csak egyszerűbbé teszik a létrehozás folyamatát, hanem elrejtik a \code{NavBar} osztály valós implementációjának részleteit. A \code{NavBarBuilder} egy lehetséges implementációját a \ref{snippet:NavBarBuilderImpl} kódcsipetben láthatjuk.

\snippet{NavBarBuilderImpl}{A NavBarBuilder egy lehetséges implementációja}