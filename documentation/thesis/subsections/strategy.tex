\subsection{Strategy}

A \emph{Strategy} viselkedési minta lehetővé teszi, hogy definiálva egy általános algoritmus interfészét, algoritmusok egész családját hozhatjuk létre, melyek mindegyike egy lehetséges, érvényes implementációját enkapszulálja az algoritmusnak, és kölcsönösen felcserélhetővé teszi őket. A minta lehetővé teszi az algoritmus interfészének és valós implementációinak elválasztását, ami azt eredményezi, hogy ezek bármikor különbözhetnek kliensektől, amelyek felhasználják, anélkül, hogy a külvilág számára észlelhető viselkedésük inkonzisztens lenne \cite{gamma1995elements}. \par

\image{Általános példa a \emph{Strategy} mintára}{strategy.drawio.png}

Az alkalmazás fejlesztése során a publikus funkcióként elérhető keresés funkció implementálásában használtuk fel. Az általános algoritmus kívánt működése az, hogy termékek egy halmazából, előre meghatározott keresési szempontok alapján meghatározza azokat a termékeket, amelyek a keresési szempontok szerint megfelelnek bizonyos kritériumoknak. \par

Ezen algoritmus köré kiépíthetünk egy strategy mintát, ahol a mintában szereplő általános algoritmust a következő interfészben definiáljuk: 

\snippet{Strategy}{Egységes interface az algoritmus elérésére}

Láthatjuk, hogy egy egységes interface mellett, amelyen keresztül elérjük az algoritmus éppen aktuális implementációját, definiáljuk azt is, hogy milyen szempontok szerint történik a keresés. Alternatív megoldás lehetett volna, ha a \code{filterSearch} nem kéri ezeket paraméterként, hanem az egyes implementáló osztályok saját maguk definiálják az általuk vizsgált szempontokat, de ez a megoldás nem lenne rugalmas. \par

 Bár így lehetséges, hogy egy implementáció olyan keresési szempontot is kap, amely számára irreleváns, az egyes implementációk egymás között teljesen felcserélhetőek. \par
 
 Jelen formájában az alkalmazás csak egyfajta keresést támogat. Akkor fogadunk el egy terméket találatként, ha:
 
 \begin{listing}
 	\item \code{name} mezőjének részsztringje a \code{properties} \code{name} mezője
 	\item \code{priceInHuf} mezőjének értéke nagyobb vagy egyenlő a \code{properties} minPrice mezőjénél
 	\item \code{priceInHuf} mezőjének értéke kisebb vagy egyenlő a \code{properties} maxPrice mezőjénél
 \end{listing}

Ezen definíció mentén a következőképpen definiáltuk az algoritmus implementációját: 

\snippet{FilterStrategy}{A SearchStrategy egy lehetséges implementációja}

Mivel a keresési szempontok is egységesítésre kerültek, így a felhasználói felület tervezésekor hagyatkozhatunk ezekre a szempontokra, egységesen vagyunk képesek validálni őket. \par

Amennyiben másfajta keresési algoritmust szeretnénk használni, például egy olyan megoldást, amiben valamilyen \emph{fuzzy search}-ön \cite{hall1980approximate} alapuló algoritmussal vizsgáljuk, hogy az éppen vizsgált elem \code{name} mezője illeszkedik-e az erre vonatkozó keresési feltételre., ezt könnyedén megtehetjük.


