\section{BCrypt}

A 90-es évek végén a mikroprocesszorok gyorsulása rohamosan növelte kibertámadások mögött álló számítási teljesítményt, míg számos autentikációs séma titkos, felhasználók által válaszott jelszavakon alapult, amelyek hossza és véletlenszerűsége közel konstans maradt. Niels Provos és David Mazières erre a problémára megoldásként egy jövőbiztos, a hardveres fejlődéssel lépést tartó jelszó sémát, és ennek részeként a BCrypt algoritmust hozta létre. \cite{provos1999future}. \par

Az algoritmust szándékosan lassúra és költségesre tervezték. Ez jó designbeli döntésnek bizonyult, mivel ezzel csökkenthető a \emph{brute-force}, azaz nyers számításierőn alapuló támadások hatékonysága. A BCrypt által nyújtott védelem tovább növelhető azzal, ha az algoritmust többször lefuttatjuk, minden új körben az előzőből kapott értéket véve hash-elendő értéknek. Ezzel a technikával még tovább inkrementálható a titkosítás hatékonysága. \par

A projektben először SHA-256 algoritmussal titkosítottuk a felhasználók jelszavát. Az emögött álló érvek a gyors, hatékony működés, illetve az alacsony erőforrásigény voltak. A későbbiekben viszont pont ezekből az okokból kifolyólag, illetve mivel hatékonysága jelentős teljesítménybeli növekedést élvez, ha GPU-n implementáljuk. Ez a tulajdonsága jelentősen növeli a brute-force alapú támadások hatékonyságát \cite{patra2021cryptography}. A BCrypt algoritmus nem rendelkezik ezzel a tulajdonsággal, nehéz hatékonyan implementálni GPU-n. Azzal, hogy egy eleve lassabb, nem párhuzamosítható algoritmust használunk, amit könnyen tudunk skálázni a több körös titkosítással, nem rontjuk sem az alkalmazást rendeltetésszerűen használó felhasználók élményét, sem az alkalmazás hatékonyságát. Amennyiben egy jogosult személy kívánja magát autentikálni, az algoritmus futási ideje szinte elhanyagolható lesz. \par