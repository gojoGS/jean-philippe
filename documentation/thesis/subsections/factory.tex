\subsection{Factory}

A \emph{single responsibility} elvéből kiindulva, minden osztálynak csak és kizárólag egy felelőséggel kell rendelkeznie \cite{martin2003agile}. Ezt figyelmebe véve el kell választanunk az objektumok példányosítását maguktól az objektumoktól. Ebben segít a \emph{Factory} minta. \par

A \emph{Factory} mintának több variánsa van. Létezik olyan verziója, amiben a létrehozandó osztály egy metódusán keresztül hozzuk létre az új objektumokat \cite{gamma1995elements}. A Spring keretrendszer bizonyos megkötései miatt egy ettől eltérő dizájnt használtunk. A Spring által nyújtott \code{@Autowired} annotációval tudjuk bizonyos objektumok létrehozásának szerepkörét átruházni a Springre. Ehhez azonban szükséges, hogy az osztály, amit példányosítani kívánunk, rendelkezzen egy argumentum nélküli konstruktorral. Vannak azonban olyan szolgáltatások, melyek működéséhez plusz információ, bizonyos paraméterek szükségesek. \par

Vegyük példaként a \code{RestaurantDetailsSetterService}-t.

\snippet{DetailsService}{RestaurantDetailsSetterService által definiált interfész}

Ez a szolgáltatás felelős az egyes éttermek tulajdonságainak megváltoztatásáért, mint például az étterem neve vagy leírása. A metódusoknak nem adjuk át paraméterként, hogy melyik étterem tulajdonságait kívánjuk megváltoztatni, hiszen így minden egyes hívásnál meg kell bizonyosodnunk róla, hogy a helyes azonosítót adjuk át. Ehelyett az implementálandó osztály felelőssége lesz ennek számontartása, egy privát mező formájában, amit egy, a konstruktorban paraméterként kapott értékkel inicializálunk. Bár ezt a megoldást is lehet helytelenül használni, kevesebbszer kell figyelmet fordítanunk a helyes használatra. Csupán egy helyen kell ügyelnünk a paraméter helyességére, továbbá, mivel ezen paraméter értéke nem változik, érdemesebb eltárolni a szolgáltatás létrehozásakor, mintsem minden metódushívásnál újra megadni. \par

Mivel a szolgáltatás konstruktora rendelkezik egy paraméterrel, nem tudjuk az \code{@Autowired} annotáció segítségével létrehozni. Viszont képesek vagyunk definiálni egy factory osztályt, aminek egy megfelelő metódusa felel az objektum létrehozásáért. \par

\snippet{Factory}{A factory interfésze}

A \code{get} metódusnak paraméterként megadunk minden kontextust a szolgáltatás létrehozásához. \par

\snippet{FactoryImpl}{Egy lehetséges factory implementáció}

A factory osztály a kontextus (azaz az étterem azonosítója) nyújtásán kívül elérést biztosít az adatelérési réteghez. A \code{RestaurantDetailsSetterService} implementációja privát belső osztályként szerepel a factory osztályban, ezzel növelve a factory osztály enkapszulációját. A factory minta ezen verziója a következő elemekből áll: \par

\begin{listing}
	\item egy létrehozandó szoftverkomponens interfésze (későbbiekben \emph{cél})
	\item az ezt létrehozó factory interfésze
	\item egy osztály (későbbiekben \emph{host}), ami implementálja a factory interfészt, illetve más szolgáltatásokat vesz igénybe; ő maga is egy szolgáltatás
	\item host privát, belső osztálya, ami implementálja célt a host által nyújtott szolgáltatások és kontextus felhasználásával.
\end{listing}





