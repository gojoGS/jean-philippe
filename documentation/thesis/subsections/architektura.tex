A leggyakoribb architekturális minta a \emph{réteg alapú architektúra}, más néven \emph{n-edfokú architektúra}. A Java alkalmazásfejlesztés de facto standardja, ezért a legtöbb mérnök és fejlesztő ismeri. Az architektúra szorosan követi a legtöbb vállalaton belüli hagyományos IT kommunikációs és szervezeti struktúrát. \par

Az architektúrán belüli komponensek vízszintes rétegekbe vannak szervezve, mindegyik egy-egy célt szolgál az alkalmazáson belül, ahogy ezt a \ref{architecture.png} ábrán is láthatjuk a saját alkalmazásunk esetében. Minden egyes réteg egy sajátos szereppel és funkcióval rendelkezik. Minden réteg egy absztrakciót formál az általa végzett szerep körül, ami egy üzleti igény teljesítéséhez szükséges. \par

Az architektúra egyik legerősebb tulajdonsága a felelősségkörök elszeparálása (angolul \emph{separation of concerns}) a komponensek között. Ez a fajta elkülönülés hatékonnyá teszi a komponensekhez tartozó szerepek definiálását és egy felelősség modell kialakítását. Az egyes komponensek jól definiált interfésze és hatásköre megkönnyíti az alkalmazásunk gyorsabb fejlesztését, tesztelését, kezelését és karbantartását \cite{richards2015software}.\par

Egy kiváló példa az alkalmazásunkból az éttermek azon felülete, ahol ételeket képesek felvenni a menübe. A felület megjelenítéséért a prezentációs réteg felel. Meghatározzuk az oldal elrendezését, stílusát, illetve funkciókat rendelünk bizonyos felhasználói inputokhoz. Az inputok feldolgozását a szolgáltatás réteg által nyújtott funkciók segítségével oldjuk meg, mely réteg már üzleti logikát tartalmaz. Ahogy a \ref{snippet:DishView} kódcsipetben láthatjuk, létrehozunk egy komponenst \code{dishCrud} néven, ami CRUD műveleteket hajt végre a paraméterként kapott szolgáltatást felhasználva. A komponens egyetlen szerepköre, hogy illeszkedjen a UI-ba, jól nézzen ki és kényelmes legyen használni. Minden egyéb feladatkör elvégzésében (például az adott ételek létrehozásában), egy alacsonyabb szintű rétegre hagyatkozik. \par

\snippet{DishView}{A UI-t definiáló osztály}

A szolgáltatás réteg egy jól definiált interfészt nyújt a külvilág számára, erre látunk példát a \ref{snippet:DishServiceInterface} kódcsipetben. Magas szintű absztrakció, ami elrejti az általa használ alacsony szintű komponensek funkcióinak, például adatbázis műveletek és hálózati kérések, komplexitását. Ahogy a \ref{snippet:DishService} csipet szemlélteti, a \code{DishService} kontextusában ez abban valósul meg, hogy magas szintű interfészt nyújt CRUD műveletekhez, míg az üzleti logika implementálásához egy alacsonyabb réteget, az adatelérési réteget használja. \par

A \code{getRestaurant} metódusban láthatjuk, amint egy adott azonosítóhoz tartozó éttermet keresünk az adatbázisban az adatelérési rétegen keresztül. Amennyiben ez a keresés sikeres, ennek eredményét visszaadja a függvény, egyébként hibát dob. \par

Üzleti logikára példát az \code{update} metódusban láthatunk, ami egy adott étterem menüjén található ételt frissít egy új állapottal, amit paraméterként kap. Az étterem ételein végzett stream műveletek segítségével megkeressük az frissítendő ételt, vagy találat hiányában visszatérünk. A \code{resultDish}-ben található példány állapotát frissítjük \code{dish}-jével, majd a változásokat elmentjük az adatbázisba a \code{restaurantRepository}-n keresztül. \par