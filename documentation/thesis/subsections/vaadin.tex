\subsection{Vaadin}

A Vaadin keretrendszerek egy családja, ami a Flow illetve a Hilla (korábban Fusion) front-end keretrendszereket foglalja magában. Az általunk használt Flow framework lehetővé teszi reszpontív, interaktív felhasználói felületek létrehozását kizárólag Java-ban, komponensek segítségével. Számos beépített komponensel és funkcióval rendelkezik, illetve számos közösség által fejlesztett plug-in-ja van. \par

A beépített komponensek között találhatóak HTML-ből ismerős elemek, mint például a

\begin{listing}
	\item \code{Input}
	\item \code{TextArea}
	\item \code{Button}
	\item \code{Div}
	\item \code{H1, H2}
	\item \code{Hr}
\end{listing}

Emellett biztosít olyan komponenseket, amelyeket nem tudunk megfeleltetni HTML elemeknek, azonban hasonló konstrukciókkal már találkozhattunk.

\begin{listing}
	\item \code{HorizontalLayout}, egy flex container, aminek \code{flex-direction} tulajdonsága \code{row} értékű 
	\item \code{VerticalLayout}, egy flex container, aminek \code{flex-direction} tulajdonsága \code{column} értékű 
	\item \code{PasswordField}, az \code{<input>} elem egy specializációja
	\item \code{NumberField}, az \code{<input>} elem egy specializációja
\end{listing}

A Vaadin lehetőség nyújt továbbá \emph{route}-ok definiálására és az ezek közötti navigációra.

\snippet{Route}{a \code{app/restaurant/staff} route-hoz tartozó nézet definíciója}

Ezek között a \code{UI} osztály \code{navigate} metódusával tudunk navigálni, úgy, hogy a metódusnak átadjuk paraméterként a cél route-hoz tartozó nézetet definiáló osztály referenciáját.

\snippet{Navigate}{Gombnyomásra visszanavigálunk a rendelésekhez}