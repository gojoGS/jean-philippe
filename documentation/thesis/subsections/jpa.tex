A Spring Boot Starter Data JPA egyszerűbbé teszi az adatbázisokkal való kommunikációt, egyszerűbbé teszi az adatelérési réteg kialakítását, mindezt oly módon, hogy a megoldás átlátható, könnyen búvíthető legyen. Amennyiben úgy döntünk, hogy nem kívánunk mi magunk adatbázis sémákat, entitásokat, lekéréseket létrehozni, hanem ezt a feladatot a Spring Data JPA-nak delegáljuk, lehetőségünk van arra, hogy a JPA hozza létre az adattáblákat, lekérdezéseket és egyéb SQL parancsokat, és ezek eredményét kezelje.

\subsubsection{Perzisztens entitások}

Ahhoz, hogy perzisztálhassuk egy osztály egy példányát egy adatbázisban 

\begin{listing}
	\item annotálnunk kell a \code{@Entity} annotációval
	\item rendelkeznie kell egy argomentumok nélküli konstruktorral
	\item minden perzisztálni kívánt mezőjének rendelkeznie kell publikus getter-el és setter-el
	\item egy mezőt meg kell jelölnünk az \code{@Id} annotációval, ami az adatbázisbeli azonosítója lesz
\end{listing}

Opcionálisan annotálhatjuk az entitás osztályt a \code{@Table} annotációval, amelynek \code{name} paramétereként megadott string lesz az entitásokat tartalmazó tábla neve. \par

Az entitás egy-egy mezőjéhez tartozó oszlop neve a mező nevével ekvivalens, amennyiben ezt nem írjuk felül a mezőn elhelyezett \code{@Column} annotáció \code{name} paraméterének adott string-gel. \par

Ügyeljünk arra, hogy semmiképen sem használjunk SQL kulcsszavakat táblák vagy mezők neveként.

\subsubsection{@Converter, AttributeConverter}

Amennyiben egy olyan értéket akarunk perzisztálni, ami nem

\begin{listing}
	\item JPA entitás
	\item JPA entitások kollekciója
	\item primitív Java típusok és ezek wrapper osztályai, illetve \code{String}
\end{listing}

\noindent biztosítanunk kell egy osztályt, ami definiál egy kölcsönös leképezést egy perzisztálható és a jelenlegi perzisztálni kívánt típus között. \par

Ennek eléréséhez definiálnunk kell egy osztályt, ami annotálva van a \code{@Converter} annotációval, illetve implementálja a \code{AttributeConverter<A, C>} generikus interfészt, ahol A az az entitás attribútum típus, amiből kiindulunk, és perzisztálni kívánjuk, C pedig az, amit valóban el tud tárolni az adatbázis. \par

Az \code{AttributeConverter<S, T>} interfésznek két definiált metódusa van:

\begin{listing}
	\item \code{C convertToDatabaseColumn(A attribute)}, ez konvertálja az entitásból kapott attribútumot egy, adatbázis által tárolható értékké
	\item \code{A convertToEntityAttribute(C column)}, ez konvertálja az adatbázis egy oszlopában szereplő értéket entitás attribútummá.
\end{listing}

Az interfészt implementáló osztályt annotálnunk kell a \code{@Converter} annotációval annak érdekében, hogy alkalmazható legyen az átváltás. Ammenyiben az annotáció \code{autoApply} paraméterének igaz értéket adunk, a perzisztancia ellátójának (angolul \emph{persistence provider}) muszály automatikusan alkalmazni a konvertert minden entitás minden perzisztált attribútumára, amire alkalmazható, kivéve, ahol az attribútumon elhelyezett \code{@Convert} annotáció ezt felülírja \cite{converterDocumentation}.

A project elkészítése során \code{enum} értékek konvertálására használtuk.

\snippet{DishType}{DishType}
\snippet{DishTypeConverter}{DishTypeConverter}

\subsubsection{@OneToOne}

\subsubsection{@OneToMany, @ManyToOne}

\subsubsection{@ManyToMany}

\subsubsection{JpaRepository, @Repository}

A Spring Data JPA által biztosított \code{JpaRepository} egyszerű, könnyen használható és kibővíthető megoldást nyújt az adatelérési réteg létrehozására. Csupán egy interface-t kell létrehoznunk, ami kiterjeszti a \code{JpaRepository<T, ID>} generikus interfészt, ahol a T a perzisztált entitás típusa, ID pedig ezen entitás azonosító mezőjének típusa \cite{jpaRepositoryDocumentation}. A \code{JpaRepository} által definiált metódusok lehetővé teszik az alapvető  DUCS adatbázis műveletek (azaz Delete, Update, Create, Select) használatát perzisztált entitásokra. \par

Amennyiben az alap műveleteken túlmutató lekérdezéseket szeretnénk definiálni, erre is van módunk. A \code{JpaRepository}-t kiterjesztő interfészben van módunk további metódusokat deklarálni, amelyek, ha nevük bizonyos szemantikai szabályokat követnek, interpretálva lesznek, mint SQL query-k. Az említett szemantikai szabályok az \ref{tab:JpaRepository} táblázatban találhatóak. \par

A \code{@Repository} annotációval jelezzük a Spring keretrendszer felé, hogy az interfész egy \emph{Repository}, azaz egy mechanizmus entitások tárolásra, kinyerésésre és keresésésre. A Spring 2.5-ös verzója óta a \code{@Component} annotáció egy specializációjaként is szolgál, azaz imlpementációi automatikusan detektálódnak.

