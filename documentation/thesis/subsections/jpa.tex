A Spring Boot Starter Data JPA egyszerűbbé teszi az adatbázisokkal való kommunikációt, egyszerűbbé teszi az adatelérési réteg kialakítását, mindezt oly módon, hogy a megoldás átlátható, könnyen búvíthető legyen. Amennyiben úgy döntünk, hogy nem kívánunk mi magunk adatbázis sémákat, entitásokat, lekéréseket létrehozni, hanem ezt a feladatot a Spring Data JPA-nak delegáljuk, lehetőségünk van arra, hogy a JPA hozza létre az adattáblákat, lekérdezéseket és egyéb SQL parancsokat, és ezek eredményét kezelje.

\subsubsection{Perzisztens entitások}

Ahhoz, hogy perzisztálhassuk egy osztály egy példányát egy adatbázisban 

\begin{listing}
	\item annotálnunk kell a \code{@Entity} annotációval
	\item rendelkeznie kell egy argomentumok nélküli konstruktorral
	\item minden perzisztálni kívánt mezőjének rendelkeznie kell publikus getter-el és setter-el
	\item egy mezőt meg kell jelölnünk az \code{@Id} annotációval, ami az adatbázisbeli azonosítója lesz
\end{listing}

Opcionálisan annotálhatjuk az entitás osztályt a \code{@Table} annotációval, amelynek \code{name} paramétereként megadott string lesz az entitásokat tartalmazó tábla neve. \par

Az entitás egy-egy mezőjéhez tartozó oszlop neve a mező nevével ekvivalens, amennyiben ezt nem írjuk felül a mezőn elhelyezett \code{@Column} annotáció \code{name} paraméterének adott string-gel. \par

Ügyeljünk arra, hogy semmiképen sem használjunk SQL kulcsszavakat táblák vagy mezők neveként.

\subsubsection{@Converter, AttributeConverter}

Amennyiben egy olyan értéket akarunk perzisztálni, ami nem

\begin{listing}
	\item JPA entitás
	\item JPA entitások kollekciója
	\item primitív Java típusok és ezek wrapper osztályai, illetve \code{String}
\end{listing}

\noindent biztosítanunk kell egy osztályt, ami definiál egy kölcsönös leképezést egy perzisztálható és a jelenlegi perzisztálni kívánt típus között. \par

Ennek eléréséhez definiálnunk kell egy osztályt, ami annotálva van a \code{@Converter} annotációval, illetve implementálja a \code{AttributeConverter<A, C>} generikus interfészt, ahol A az az entitás attribútum típus, amiből kiindulunk, és perzisztálni kívánjuk, C pedig az, amit valóban el tud tárolni az adatbázis. \par

Az \code{AttributeConverter<S, T>} interfésznek két definiált metódusa van:

\begin{listing}
	\item \code{C convertToDatabaseColumn(A attribute)}, ez konvertálja az entitásból kapott attribútumot egy, adatbázis által tárolható értékké
	\item \code{A convertToEntityAttribute(C column)}, ez konvertálja az adatbázis egy oszlopában szereplő értéket entitás attribútummá.
\end{listing}

Az interfészt implementáló osztályt annotálnunk kell a \code{@Converter} annotációval annak érdekében, hogy alkalmazható legyen az átváltás. Ammenyiben az annotáció \code{autoApply} paraméterének igaz értéket adunk, a perzisztancia ellátójának (angolul \emph{persistence provider}) muszály automatikusan alkalmazni a konvertert minden entitás minden perzisztált attribútumára, amire alkalmazható, kivéve, ahol az attribútumon elhelyezett \code{@Convert} annotáció ezt felülírja \cite{converterDocumentation}.

A project elkészítése során \code{enum} értékek konvertálására használtuk.

\snippet{DishType}{DishType}
\snippet{DishTypeConverter}{DishTypeConverter}

\subsubsection{JPA kapcsolatok}

A JPA kapcsolatoknak két fajtája van:
\begin{listing}
	\item egyirányú (\emph{unidirectional})
	\item kétirányú (\emph{bidirectional})
\end{listing}

Két entitás közötti kapcsolat definiálásának nincs hatása az entitások adatbázisra való leképezésének, csupán azt definiálja, hogy milyen irányban tudjuk használni a kapcsolatot a domain modellünkben. \par

Objektumok esetében a kétirányú kapcsolat egy definiált fogalom, szemantikája jól meghatározható, a relációs adatbázisok esetében azonban ez a fogalom nem létezik, csupán egyirányú kapcsolatok vannak (foreign key-ek formájában). Így a Hibernate, ami az objektumok relációkra való leképzést végzi, két egyirányú kapcsolattal modellezi a kétirányú kapcsolatot. A Hibernate a kapcsolat mindkét oldalának változásait követi, így például ha a kapcsolat pontosan egy résztvevője megváltozik, az az adatbázisban is meg fog változni, de ez a változás nem fog manifesztálódni a kapcsolat másik oldalán, azaz inkonzisztencia lép fel az adatbázisban. Ezt a problémát oldja meg a birtokló oldal, birtoklott oldal meghatározása. \par

Ha Hibernate csupán a birtokló oldal változásait követi nyomon, így ha a birtokló oldalon bármilyen változás történik a kapcsolatban, azt képes leképezni a kapcsolat másik felére, a kapcsolat birtokolt oldalán történő változtatásokat pedig nem követi. Ez a megoldás megoldja a kétirányú kapcsolatokból eredő inkonzisztenciákat. A birtokolt oldal könnyen felismerhető onnan, hogy értéket adunk a kapcsolatot jelző annotáció \code{mappedBy} paraméterének. \par

\subsubsection{@OneToOne}

Amennyiben egy mezőt a \code{@OneToOne} annotációval látunk el, egy egy-az-egyhez leképezést tudunk létrehozni két entitás között, azaz egy olyan leképezést, amelyben egy entitás egy példányához egy másik entitás legfeljebb egy példányát rendeljük. Az egy-az-egyhez leképezést több módon implementálhatjuk:

\begin{listing}
	\item \emph{foreign key} használatával
	\item \emph{közös primary key} használatával
	\item \emph{join table} használatával
\end{listing}

A project elkészítése során foreign key használatával implementáltuk a kapcsolatot. Ebben az implementációban a \code{@OneToOne}-nal annotált mezőt a \code{@JoinColumn} annotációval is el kell látni, amelynek \code{name} paraméterével adjuk meg a foreign key-t tartalmazó oszlop nevét. Ez az oszlop a kapcsolatban szereplő birtokló fél táblájában fog szerepelni. \par

Amennyiben a kapcsolat egyirányú, a birtokló fél egyértelműen meghatározható: az az entitás, amely a \code{@OneToOne}-nal annotált mezőt tartalmazza.\par

Ha a kapcsolatot kétirányúvá szeretnénk bővíteni, a egy-az-egyhez kapcsolat másik felét képző entitásban el kell látnunk a kapcsolat másik felének mezőjét a \code{@OneToOne} annotációval, amelynek \code{mappedBy} paraméterével adjuk meg, hogy a másik oldal melyik mezője hivatkozik rá. \par

\snippet{OneToOne}{RestaurantTable mint birtokló, EndUser, mint birtokolt}

\subsubsection{@OneToMany, @ManyToOne}
A \code{@OneToMany} és a \code{@ManyToOne} annotációkkal egy-a-többhöz és több-az-egyhez kapcsolatokat tudunk definiálni. Amennyiben külön-külön alkalmazzuk őket, ezek egyirányú kapcsolatokat határoznak meg, azonban ha egymással párban, két entitáson, akkor egy darab, kétirányú kapcsolatot definiál az entitások között. \par

A projectben ezzel a két annotációval reprezentáltuk a rendelések és a pincérek közötti kapcsolatot, azaz egy pincérnek több, hozzá rendelt rendelése lehet, míg egy rendeléshez legfeljebb egy pincér rendelhető.

\snippet{OneToMany}{}

\subsubsection{@ManyToMany}

A project során a rendelések és a rendelésekben szereplő termékek közötti kapcsolatot több-a-többhoz kapcsolattal írtuk le, mivel egy rendeléshez több termék tartozik, illetve egy termék több rendeléshez is hozzárendelhető. \par

Több féle módon implementálhatjuk: 

\begin{listing}
	\item \emph{join table} használatával
	\item egy új, közvetítő entitás létrehozásával, ami enkapszulálja a kapcsolatban résztvevő feleket (például egy ItemsOfOrder entitás, ami tartalmazná hogy melyik rendelés mely termékkel van asszociálva)
	\item \emph{composite key} használatával
\end{listing}

A project során \emph{join table}-t felhasználva implementáltuk. A birtokló oldalon definiálni kell a kapcsolatokat tartalmazó tábla nevét, illetve hogy a kapcsolatban résztvevő felek id-jai ennek a táblának mely oszlopaiban szerepelnek \par

\snippet{ManyToMany}{Order, mint birtokló, Item, mint birtokolt}

\subsubsection{JpaRepository, @Repository}

A Spring Data JPA által biztosított \code{JpaRepository} egyszerű, könnyen használható és kibővíthető megoldást nyújt az adatelérési réteg létrehozására. Csupán egy interface-t kell létrehoznunk, ami kiterjeszti a \code{JpaRepository<T, ID>} generikus interfészt, ahol a T a perzisztált entitás típusa, ID pedig ezen entitás azonosító mezőjének típusa \cite{jpaRepositoryDocumentation}. A \code{JpaRepository} által definiált metódusok lehetővé teszik az alapvető  DUCS adatbázis műveletek (azaz Delete, Update, Create, Select) használatát perzisztált entitásokra. \par

Amennyiben az alap műveleteken túlmutató lekérdezéseket szeretnénk definiálni, erre is van módunk. A \code{JpaRepository}-t kiterjesztő interfészben van módunk további metódusokat deklarálni, amelyek, ha nevük bizonyos szemantikai szabályokat követnek, interpretálva lesznek, mint SQL query-k. Az említett szemantikai szabályok az \ref{tab:JpaRepository} táblázatban találhatóak. \par

A \code{@Repository} annotációval jelezzük a Spring keretrendszer felé, hogy az interfész egy \emph{Repository}, azaz egy mechanizmus entitások tárolásra, kinyerésésre és keresésésre. A Spring 2.5-ös verzója óta a \code{@Component} annotáció egy specializációjaként is szolgál, azaz imlpementációi automatikusan detektálódnak.

