\subsection{Lombok}

A Lombok könyvtár, aminek célja a repetatív kódrészletek (úgynevezett boilerplate kód) írásának elkerülése, a fejlesztői élmény javítása. A legtöbb esetben nem terjeszti ki a Java funkcióit, hanem már meglévő funkciók használatát teszi kényelmesebbé. \par

Bizonyos keretrendszerek disziplínái megkövetelik például a getter-ek és setter-ek, bizonyos contructor-ok definiálását. Ilyen esetben hasznlhatjuk a Lombok, ami a build folyamatunkba beépülve, valid Java bytecode-ot generál automatikusan azokban az osztályokban, ahol bevezettük az annotációit.\par

Amennyiben kíváncsiak vagyunk arra, hogy a Lombok milyen transzformációkat hajt végre a kódunkon, vagy egyszerűen csak meg akarunk válni tőle, és eltávolítani a függőségink közül, a kódbázisunkat pedig megtisztítani a Lombok annotációktól, abban az esetben erre is van lehetőség. \par

A delombok nevű eszköz előállítja azokat az osztályok forráskódját, amikben Lombok annotációt használtunk, eltávolítva az annotációkat és helyükre a velük ekvivalens kód kerül, amelyet eddig a Lombok generált. \par


\subsubsection{@NoArgsConstructor, @AllArgsConstructor}

A \lstinline|@NoArgsConstructor|-t egy osztályra helyezve egy argomentumok nélküli konstruktort fog generálni. Amennyiben ez nem lehetséges, például egy \lstinline|final| mező miatt, a generálási folyamat egy fordítási hibát fog eredményezni \cite{lombokConstructorDocumentation}. Ez megkerülhető úgy, ha az annotáció \lstinline|force| paraméterének \lstinline|true| értéket adunk meg. Ezzel elérjük, hogy a \lstinline|final| mezők is inicializálva lesznek \lstinline|0|, \lstinline|false| vagy \lstinline|null| értékkel, azonban olyan mezők esetében, a \lstinline|@NonNull| annotációval végzünk null vizsgálatot, ez a vizsgálat nem fog legenerálódni, megtörténni. \par

\snippet{NoArgsConstructorBefore}{@NoArgsConstructor}
\snippet{NoArgsConstructorAfter}{@NoArgsConstructor delombok után}

A project elkészítése során Spring Data JPA entitás osztályokat láttunk el \lstinline|@NoArgsConstructor| annotációval, mivel a JPA megköveteli egy ilyen konstruktor létezését. \par


A \lstinline|@AllArgsConstructor| egy konstruktort generál egy osztálynak, annak egy-egy mezőjéhez tartozó egy paraméterrel. Amennyiben egy mező el van látva a \lstinline|@NonNull| annotációval, a generált konstruktor egy null check-et fog végezni azon a mezőn.

\snippet{AllArgsConstructorBefore}{@AllArgsConstructor}
\snippet{AllArgsConstructorAfter}{@AllArgsConstructor delombok után}

Az \code{@AllArgsConstructor} hátránya, hogy csak azok a mezők szerepelnek a konstruktor paraméterei között, amelyeket az osztályban deklaráltunk, azaz a szülő osztály mezőihez tartozó argomentumok nem szerepelnek a gyermek osztály generált konstruktorában. Ennek ellenére használata eredményes volt a projectben, például DTO-k és adattagokkal rendelkező enumok definniálásánál.


\subsubsection{@Getter, @Setter}

Egy osztály bármilyen mezőjét annotálhatjuk a  \lstinline|@Getter|-rel vagy \lstinline|@Setter|-rel, a Lombok automatikusan generálni fog egy alapértelmezett getter/setter metódust. \par

A \code{foo} mező alapértelmezett getter-je a \code{getFoo} nevű metódus, illetve \code{boolean} típusú mező esetében \code{isFoo}, ami a \code{foo} mezőt téríti vissza. Ezen mező alapértelmezett setter-je egy \code{setFoo} nevű, egy paraméteres metódus, amely paraméter típusa azonos a mező típusával. \par

A generált metódus publikus láthatóságú lesz, amennyiben ezt nem írjuk felül az annotációban elhelyezett \code{AccessLevel} értékkel. Ennek lehetségek értékei \code{PUBLIC}, \code{PROTECTED}, \code{PACKAGE} és \code{PRIVATE}, melyek rendre a Java nyelv láthatóságainak felelelnek meg. \par

Ezeket az annotációkat nemcsak mezőkön, hanem osztályokon is alkalmazhatjuk. Ebben az esetben az annotációhatása ekvivalens azzal, h az osztály minden, nem statikus mezőjét annotáltuk volna. Kivételt képeznek azok a mezők, amelyeket menuálisan annotálunk és láthatóságnak \code{AccessLevel.NONE}-t határozunk meg. \par

\snippet{GetterSetterBefore}{@Getter, @Setter}
\snippet{GetterSetterAfter}{@Getter, @Setter delombok után}

\subsubsection{@ToString}

Bármely osztály annotálható \code{@ToString}-el, amely felülírja az osztály \code{toString} metódusát, egy, a Lombok által generált implementációval. Alapértelmezetten, ezen implementáció által visszaadott string tartalmazza az osztály nevét, követve a az osztály nem statikus mezőinek nevével és ezek értékeivel, a deklarálásuk sorrendjében. \par

Amenability nem akarjuk, hogy minden mező megjelenjen a metódus output-jában, jelezhetjük a nem kívánt mezőket a \code{@ToString.Exclude} annotációval, felsorolhatjuk a tartalmazni kívánt mezők nevét a \code{@ToString} metódus \code{includeFieldNames} paraméterében, vagy a \code{onlyExplicitlyIncluded} paraméternek megadott \code{true} értékkel és a tartalmazni kívánt mezőkön elhelyezett \code{@ToString.Include} annotációval. Amennyiben a \code{callSuper} praméternek \code{true} értéket adunk, a visszaadott string tartalmazni fogja a szülő osztály \code{toString} metódusának output-ját \par

\subsubsection{@Builder}

A \code{@Builder} annotációval annotált \code{Foo} osztályhoz, konstruktorhoz vagy metódushoz (a későbbiekben ez \emph{target}) generálódik egy belső, statikus, \code{FooBuilder} nevű osztály (a későbbiekben ez \emph{builder}). \par

A builder osztály tartalmaz target egy-egy paraméteréhez vagy mezőjéhez tartozó, privát, nem statikus, nem \code{final} mezőt, egy package private no-args konstruktort. Builder minden mezője rendelkezik egy setter-szerű metódussal, ami a mező értékéhez a paraméterként kapott értéket rendeli, és a builder példányt adja vissza, ezzel elhetővé téve a metódushívások egymásba láncolását. Builder rendelkezik egy \code{build} metódussal, amely meghívásakor meghívódik target (amennyiben target osztály, annak egy megfelelő konstruktora), builder mezőinek értékével, és ugyanazt a típust adja vlssza, mint target (osztály esetében ez target). \par

A target-et tartalmazó osztályban (amennyiben target osztály, ez target)  generálódik egy \code{builder} metódus, ami builder egy új példányát hozza létre. \par

\snippet{BuilderBefore}{@Builder}
\snippet{BuilderAfter}{@Builder delombok után}

\subsubsection{@Slf4j}

A \code{@Log} annotáció számos variánssal rendelkezik. Ezen variánsok egy-egy log kezelő megoldáshoz készülnek, mivel számos ilyen megoldás van a JVM platformon, egy annotáció családról beszéhetünk. Közös tulajdonsága a család tagjainak, hogy az annotáció elhelyezése után elérhető egy \code{log} nevű objektum, ami az adott logolási megoldás logger példánya. Az \code{@Slf4j} az azonos nevű loggert teszi elérhetővé. \par

A fejlesztés során hasznosnak bizonyult, mivel a kódbázis tisztább és kisebb lett, a kód kevesebb zajt tartalmaz.

\snippet{Slf4jBefore}{@Slf4j}
\snippet{Slf4jAfter}{@Slf4j delombok után}



