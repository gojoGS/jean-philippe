\subsection{UUID}

Az univerzálisan egyedi azonosító (angolul \emph{Universally unique identifier}) vagy UUID egy ITF szabvány, amit az RFC 4122 definiált. A fő motiváció használatára az, hogy nem szükséges egy központi szerv bevonása a létrejövő azonosítók adminisztrálására, így generálásuk teljes mértékben automatizálható. A UUID-k fix mérete miatt, ami 128 bit, jelentősen kisebb, mint a legtöbb alternatív megoldás. Kompakt méretéből következik, hogy használata optimálisabb teljesítményt eredményez rendező és hasító algoritmusok, illetve adatbázisban való tárolás esetében. Az említett RFC-ben leírt generáló algoritmus akár másodpercenként 10 millió allokációt tud elvégezni gépenként, melyből kifolyólag a UUID-t tranzakciók azonosítójaként is használhatjuk. \par

A UUID azonosítók valójában nem \emph{teljesen} egyediek, azaz van esély arra, hogy két generált azonosító ugyanazzal az értékkel fog rendelkezni, ütközni fognak. Azonban ennek a valószínűsége elenyésző. A UUID ütközést tekinthetjük a születésnap probléma egy speciális esetének. Annak az esélye, hogy egy populációban, egymástól függetlenül kiosztott $x$ azonosító közül $n$-et kiválasztva $p$ valószínűséggel legyen köztük legalább kettő egyező az

\begin{equation*}
	n = 0.5 + \sqrt{0.25 - 2\times(\ln q)\times x}
\end{equation*}

formulával kiválóan lehet közelíteni \cite{mathis1991generalized}, ahol $q = 1 - p$. Ebből adódóan ahhoz, hogy legalább 50\%-os eséllyel generálódjon legalább két UUID,

\begin{equation*}
	n \approx 0.5 + \sqrt{0.25 + 2\times(\ln 2)\times2^{122}} \approx 2.71 \times 10^{18}
\end{equation*}

azonosítót kellene generálnunk. Ehhez, a másodpercenkénti 10 milló generált azonosítóból kiindulva megközelítóleg 8587 év szükséges, azaz ennyi idő szükséges ahhoz, hogy 50\%-os valószínűséggel előidézzünk egy UUID ütközést. \par

Egy étterem egy, a rendszerbe felvett asztalához tartozó, automatikusan létrehozott végfelhasználói fiók "felhasználóneve" egy UUID, amit az említett fiók létrehozásakor generálunk. Ez egy optimális megoldás, hiszen az étteremnek nem kell "kitalálni" valamilyen egyedi azonosítót a végfelhasználói fióknak, így gyorsabb és felhasználóbarátabb az asztalok a rendszerbe való felvezetésének folyamata. A UUID tulajdonságaiból kiindulva ezek az azonosítók ésszerű keretek között valóban egyediek lesznek.