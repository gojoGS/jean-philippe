
	\image{Alkalmazás architektúra}{architecture.png}
	\image{Regisztráció}{jp-signup.png}
	\image{Menü}{jp-login.png}
	\image{Menü}{rs-menu.png}
	\image{Ital létrehozása}{rs-add-beverage.png}
	\image{Profil}{rs-profile.png}
	\image{Asztalok}{rs-tables.png}
	\image{Végfelhasználók}{rs-end-users.png}
	\image{Új jelszó}{rs-new-password.png}
	\image{Személyzet}{rs-staff.png}
	\image{Végfelhasználói kezdőképernyő}{eu-start.png}
	\image{A rendelés}{eu-menu-ready.png}
	\image{Várakozó képernyő}{eu-sent.png}
	\image{Új rendelés érkezett}{rs-new-order.png}
	\image{Rendelések}{rs-orders.png}
	\image{Az étterem elfogadta a rendelést}{eu-updated.png}
	\image{Várakozás a rendelés elkészülésére}{eu-waiting.png}
	\image{Rendelés lezárása}{rs-close-order.png}
	\image{Fizetési szándék jelzése}{eu-check-out.png}
	
	\footnotesize
	\begin{longtable}{| p{.20\linewidth} | p{.30\linewidth} | p{.50\linewidth } |}		
			\caption{JpaRepository kiterjesztéséhez  használható kulcsszavak}
			\label{tab:JpaRepository} \\
			\hline
			Kulcsszó & Minta & JPQL kódcsipet \\ 
			\hline
			And &	findByLastnameAndFirstname	& … where x.lastname = ?1 and x.firstname = ?2 \\
			\hline
			Or &	findByLastnameOrFirstname	& … where x.lastname = ?1 or x.firstname = ?2 \\
			\hline
			Is,Equals &	findByFirstname \linebreak findByFirstnameIs \linebreak findByFirstnameEquals	& … where x.firstname = 1? \\
			\hline
			Between &	findByStartDateBetween	& … where x.startDate between 1? and ?2 \\
			\hline
			LessThan &	findByAgeLessThan	& … where x.age < ?1 \\
			\hline
			LessThanEqual &	findByAgeLessThanEqual	& … where x.age <= ?1 \\
			\hline
			GreaterThan &	findByAgeGreaterThan	& … where x.age > ?1 \\
			\hline
			GreaterThanEqual &	findByAgeGreaterThanEqual	& … where x.age >= ?1 \\
			\hline
			After &	findByStartDateAfter	& … where x.startDate > ?1 \\
			\hline
			Before &	findByStartDateBefore	& … where x.startDate < ?1 \\
			\hline
			IsNull &	findByAgeIsNull	& … where x.age is null \\
			\hline
			IsNotNull,NotNull &	findByAge(Is)NotNull	& … where x.age not null \\
			\hline
			Like &	findByFirstnameLike	& … where x.firstname like ?1 \\
			\hline
			NotLike &	findByFirstnameNotLike	& … where x.firstname not like ?1 \\
			\hline
			StartingWith &	findByFirstnameStartingWith	& … where x.firstname like ?1 (parameter bound with appended \%) \\
			\hline
			EndingWith &	findByFirstnameEndingWith	& … where x.firstname like ?1 (parameter bound with prepended \%) \\
			\hline
			Containing &	findByFirstnameContaining	& … where x.firstname like ?1 (parameter bound wrapped in \%) \\
			\hline
			OrderBy &	findByAgeOrderByLastnameDesc	& … where x.age = ?1 order by x.lastname desc \\
			\hline
			Not &	findByLastnameNot	& … where x.lastname <> ?1 \\
			\hline
			In &	findByAgeIn(Collection<Age> \linebreak ages)	& … where x.age in ?1 \\
			\hline
			NotIn &	findByAgeNotIn(Collection<Age> \linebreak age)	& … where x.age not in ?1 \\
			\hline
			True &	findByActiveTrue()	& … where x.active = true \\
			\hline
			False &	findByActiveFalse()	& … where x.active = false \\
			\hline
			IgnoreCase &	findByFirstnameIgnoreCase	& … where UPPER(x.firstame) = UPPER(?1) \\
			\hline
	\end{longtable}

	\begin{longtable}{| p{.20\linewidth} | p{.50\linewidth} | p{.30\linewidth } |}		
					\caption{HTTP kérések és szerepük REST szolgáltatásokban}
		\label{tab:restAndHttp} \\
		\hline
		HTTP kérés & Szerepe & HTTP válasz kód \\
		\hline
		
		\multirow{3}{*}{GET} & \multirow{3}{*}{erőforrás reprezentációnak kinyerése} & 
		 200: az erőforrás megtalálható a szerveren \\
		 \cline{3-3}
		  & & 404: az erőforrás nem található meg \\
		  		 \cline{3-3}
		  & & 400: a kérés formázása nem megfelelő \\
		\hline
		
		\multirow{2}{*}{POST} & \multirow{2}{*}{új erőforrás létrehozása} & 
		201: az erőforrás létrejött a szerveren\\
		\cline{3-3}
		& & 200, 204: a létrejött erőforrás nem azonosítható URI-val \\
		\hline
		
		\multirow{2}{*}{PUT} & \multirow{2}{*}{\parbox{\linewidth}{létező erőforrás állapotának frissítése; ha az nem létezik a  szolgáltatás dönthet létrehozásáról}} & 
		201: új erőforrás jött létre \\
		\cline{3-3}
		& & 200, 204: egy létező erőforrás frissítése sikeres volt \\
		\hline
		
		\multirow{3}{*}{DELETE} & \multirow{3}{*}{erőforrás törlése} & 
		200: az erőforrás törölve lett \\
		\cline{3-3}
		& & 202: a kérés elfogadásra került, feldolgozásra vár \\
		\cline{3-3}
		& & 204: az erőforrás törölve lett, státuszáról nem érkezett információ a válaszban \\
		\hline
		
		PATCH & egy erőforrás részleges frissítése & nincs ajánlás\\
		\hline
	\end{longtable}