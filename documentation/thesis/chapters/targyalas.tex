\section{A program struktúrája, bemutatása}

\section{Tervezési minták}

\subsection{Factory}

factory

\subsection{Builder}

builder

\subsection{Adapter}

adapter

\subsection{Strategy}

A \emph{Strategy} viselkedési minta lehetővé teszi, hogy definiálva egy általános algoritmus interfészét, algoritmusok egész családját vagyunk képesek létrehozni, melyek mindegyike egy lehetséges, érvényes implementációját enkapszulálja az algoritmusnak, és kölcsönösen felcserélhetővé teszi őket. A minta lehetővé teszi az algoritmus interfészének és valós implementációinak elválasztását, ami azt eredményezi, hogy ezek bármikor különbözhetnek kliensektől, amelyek felhasználják, anélkül, hogy a külvilág számára észlelhető viselkedésük inkonzisztens lenne. \par



\section{Használt könyvtárak és keretrendszerek}

A lombok könyvtár, aminek célja a repetatív kódrészletek (úgynevezett boilerplate kód) írásának elkerülése, a fejlesztői élmény javítása. A legtöbb esetben nem terjeszti ki a Java funkcióit, hanem már meglévő funkciók használatát teszi kényelmesebbé. \par

Bizonyos keretrendszerek disziplínái megkövetelik például a getter-ek és setter-ek, bizonyos contructor-ok definiálását. Ilyen esetben hasznlhatjuk a lombokot, ami a build folyamatunkba beépülve, valid Java bytecode-ot generál automatikusan azokban az osztályokban, ahol bevezettük az annotációit.\par

Amennyiben kíváncsiak vagyunk arra, hogy a lombok milyen transzformációkat hajt végre a kódunkon, vagy egyszerűen csak meg akarunk válni tőle, és eltávolítani a függőségink közül, a kódbázisunkat pedig megtisztítani a lombok annotációktól, abban az esetben erre is van lehetőség. \par

A delombok nevű eszköz előállítja azokat az osztályok forráskódját, amikben lombok annotációt használtunk, eltávolítva az annotációkat és helyükre a velük ekvivalens kód kerül, amelyet eddig a lombok generált.


\subsubsection{@NoArgsConstructor}

A \lstinline|@NoArgsConstructor|-t egy osztályra helyezve egy argomentumok nélküli konstruktort fog generálni. Amennyiben ez nem lehetséges, például egy \lstinline|final| mező miatt, a generálási folyamat egy fordítási hibát fog eredményezni \cite{lombokConstructorDocumentation}. Ez megkerülhető úgy, ha az annotáció \lstinline|force| paraméterének \lstinline|true| értéket adunk meg. Ezzel elérjük, hogy a \lstinline|final| mezők is inicializálva lesznek \lstinline|0|, \lstinline|false| vagy \lstinline|null| értékkel, azonban olyan mezők esetében, a \lstinline|@NonNull| annotációval végzünk null vizsgálatot, ez a vizsgálat nem fog legenerálódni, megtörténni. 

\lstinputlisting[language=Java]{code/NoArgsConstructorBefore.java}

\lstinputlisting[language=Java]{code/NoArgsConstructorAfter.java}

A project elkészítése során Spring Data JPA entitás osztályokat láttunk el \lstinline|@NoArgsConstructor| annotációval, mivel a JPA megköveteli egy ilyen konstruktor létezését.

\subsubsection{@AllArgsConstructor}

A \lstinline|@AllArgsConstructor| egy konstruktort generál egy osztálynak, mezőinként egy paraméterrel. Amennyiben egy mező el van látva a \lstinline|@NonNull| annotációval, a generált konstruktor egy null check-et fog végezni azon a mezőn.

\lstinputlisting[language=Java]{code/AllArgsConstructorBefore.java}

\lstinputlisting[language=Java]{code/AllArgsConstructorAfter.java}

\subsubsection{@Getter, @Setter}

\subsubsection{@ToString}

\subsubsection{@Builder}

\subsubsection{@Slf4j}






\subsection{Spring Boot}

boot

A Spring Boot Starter Data JPA egyszerűbbé teszi az adatbázisokkal való kommunikációt, egyszerűbbé teszi az adatelérési réteg kialakítását, mindezt oly módon, hogy a megoldás átlátható, könnyen búvíthető legyen. Amennyiben úgy döntünk, hogy nem kívánunk mi magunk adatbázis sémákat, entitásokat, lekéréseket létrehozni, hanem ezt a feladatot a Spring Data JPA-nak delegáljuk, lehetőségünk van arra, hogy a JPA hozza létre az adattáblákat, lekérdezéseket és egyéb SQL parancsokat, és ezek eredményét kezelje.

\subsubsection{Perzisztens entitások}

Ahhoz, hogy perzisztálhassuk egy osztály egy példányát egy adatbázisban 

\begin{listing}
	\item annotálnunk kell a \code{@Entity} annotációval
	\item rendelkeznie kell egy argomentumok nélküli konstruktorral
	\item minden perzisztálni kívánt mezőjének rendelkeznie kell publikus getter-el és setter-el
	\item egy mezőt meg kell jelölnünk az \code{@Id} annotációval, ami az adatbázisbeli azonosítója lesz
\end{listing}

Opcionálisan annotálhatjuk az entitás osztályt a \code{@Table} annotációval, amelynek \code{name} paramétereként megadott string lesz az entitásokat tartalmazó tábla neve. \par

Az entitás egy-egy mezőjéhez tartozó oszlop neve a mező nevével ekvivalens, amennyiben ezt nem írjuk felül a mezőn elhelyezett \code{@Column} annotáció \code{name} paraméterének adott string-gel. \par

Ügyeljünk arra, hogy semmiképen sem használjunk SQL kulcsszavakat táblák vagy mezők neveként.

\subsubsection{@Converter, AttributeConverter}

Amennyiben egy olyan értéket akarunk perzisztálni, ami nem

\begin{listing}
	\item JPA entitás
	\item JPA entitások kollekciója
	\item primitív Java típusok és ezek wrapper osztályai, illetve \code{String}
\end{listing}

\noindent biztosítanunk kell egy osztályt, ami definiál egy kölcsönös leképezést egy perzisztálható és a jelenlegi perzisztálni kívánt típus között. \par

Ennek eléréséhez definiálnunk kell egy osztályt, ami annotálva van a \code{@Converter} annotációval, illetve implementálja a \code{AttributeConverter<A, C>} generikus interfészt, ahol A az az entitás attribútum típus, amiből kiindulunk, és perzisztálni kívánjuk, C pedig az, amit valóban el tud tárolni az adatbázis. \par

Az \code{AttributeConverter<S, T>} interfésznek két definiált metódusa van:

\begin{listing}
	\item \code{C convertToDatabaseColumn(A attribute)}, ez konvertálja az entitásból kapott attribútumot egy, adatbázis által tárolható értékké
	\item \code{A convertToEntityAttribute(C column)}, ez konvertálja az adatbázis egy oszlopában szereplő értéket entitás attribútummá.
\end{listing}

Az interfészt implementáló osztályt annotálnunk kell a \code{@Converter} annotációval annak érdekében, hogy alkalmazható legyen az átváltás. Ammenyiben az annotáció \code{autoApply} paraméterének igaz értéket adunk, a perzisztancia ellátójának (angolul \emph{persistence provider}) muszály automatikusan alkalmazni a konvertert minden entitás minden perzisztált attribútumára, amire alkalmazható, kivéve, ahol az attribútumon elhelyezett \code{@Convert} annotáció ezt felülírja \cite{converterDocumentation}.

A project elkészítése során \code{enum} értékek konvertálására használtuk.

\snippet{DishType}{DishType}
\snippet{DishTypeConverter}{DishTypeConverter}

\subsubsection{@OneToOne}

\subsubsection{@OneToMany, @ManyToOne}

\subsubsection{@ManyToMany}

\subsubsection{JpaRepository, @Repository}

A Spring Data JPA által biztosított \code{JpaRepository} egyszerű, könnyen használható és kibővíthető megoldást nyújt az adatelérési réteg létrehozására. Csupán egy interface-t kell létrehoznunk, ami kiterjeszti a \code{JpaRepository<T, ID>} generikus interfészt, ahol a T a perzisztált entitás típusa, ID pedig ezen entitás azonosító mezőjének típusa \cite{jpaRepositoryDocumentation}. A \code{JpaRepository} által definiált metódusok lehetővé teszik az alapvető  DUCS adatbázis műveletek (azaz Delete, Update, Create, Select) használatát perzisztált entitásokra. \par

Amennyiben az alap műveleteken túlmutató lekérdezéseket szeretnénk definiálni, erre is van módunk. A \code{JpaRepository}-t kiterjesztő interfészben van módunk további metódusokat deklarálni, amelyek, ha nevük bizonyos szemantikai szabályokat követnek, interpretálva lesznek, mint SQL query-k. Az említett szemantikai szabályok az \ref{tab:JpaRepository} táblázatban találhatóak. \par

A \code{@Repository} annotációval jelezzük a Spring keretrendszer felé, hogy az interfész egy \emph{Repository}, azaz egy mechanizmus entitások tárolásra, kinyerésésre és keresésésre. A Spring 2.5-ös verzója óta a \code{@Component} annotáció egy specializációjaként is szolgál, azaz imlpementációi automatikusan detektálódnak.



\subsection{Vaadin}

vaadin

\section{Felhasznált szabványok, ajánlások}

\subsection{UUID}

Az univerzálisan egyedi azonosító (angolul \emph{Universally unique identifier}) vagy UUID egy ITF szabvány, amit az RFC 4122 definiált. A fő motiváció használatára az, hogy nem szükséges egy központi szerv bevonása a létrejövő azonosítók adminisztrálására, így generálásuk teljes mértékben automatizálható. A UUID-k fix mérete miatt, ami 128 bit, jelentősen kisebb, mint a legtöb alternatív megoldás. Kompakt méretéből következik, hogy használata optimálisabb teljesítményt eredményez rendező és hasító algoritmusok, illetve adatbázisban való tárolás esetében. Az említett RFC-ben leírt generáló algoritmus akár másodpercenként 10 milló allokációt tud elvégezni gépenként, melyből kifolyólag a UUID-t tranzakciók azonosítójaként is használható. \par

A UUID azonosítók valójában nem \emph{teljesen} egyediek, azaz van esély arra, hogy két generált azonosító ugyanazzal az értékkel fog rendelkezni, ütközni fognak. Azonban ennek a valószínűsége elenyésző. A UUID ütközést tekinthetjük a születésnap probléma egy speciális esetének. Annak az esélye, hogy egy populációban, egymástól függetlenül kiosztott $x$ azonosító közül $n$-et kiválasztva $p$ valószínűséggel legyen köztük legalább kettő egyező az

\begin{equation*}
	n = 0.5 + \sqrt{0.25 - 2\times(\ln q)\times x}
\end{equation*}

formulával kiválóan lehet közelíteni \cite{mathis1991generalized}, ahol $q = 1 - p$. Ebből adódóan ahhoz, hogy legalább 50\%-os eséllyel generálódjon legalább két UUID,

\begin{equation*}
	n \approx 0.5 + \sqrt{0.25 + 2\times(\ln 2)\times2^{122}} \approx 2.71 \times 10^{18}
\end{equation*}

azonosítót kellene generálnunk. Ehhez, a másodpercenkénti 10 milló generált azonosítóból kiindulva megközelítóleg 8587 év szükséges, azaz ennyi idő suükséges ahhoz, hogy 50\%-os valószínűséggel előidézzükn egy UUID ütközést.

\subsection{REST API}

Az állapotreprezentáció-transzfer (angolul \emph{representational state transfer}) vagy REST nem egy konkrét, jól definiált standard, sokkal inkább egy architektúrális stílus, ami elterjedt technológiákat és szabványokat használ fel web alapú szolgáltatások tervezésére és implementálására \cite{richards2006representational}. \par

\emph{Roy Thomas Fielding} doktori disszertációban számos megkötést tett arra, hogy hogyan definiálható egy REST szolgáltatás architektúrája \cite{fielding2000architectural}. 

\subsubsection{Kliens-szerver architektúra}
Emögött a \emph{separation-of-concerns} elv áll. Azzal, hogy elválasztjuk a felhasználói interfészt az adattárolás szerepkörétől, skálázhatóbbá és szélesebb körben portolhatóvá a felhasználói interfész. Emellett ez az elkülönülés megengedi, hogy külön a szerverkomponensek egymástól elkülönülve fejlődjenek.

\subsubsection{Állapotmentes}
A kliens és szerver közti kommunikációnak állapotmentesnek kell lennie, oly módon, hogy a klienstől érkező minden kérése tartalmazza az összes, a kérés feldolgozásához szükséges információt, nem hagyatkozhat bármilyen, a szerveren tárolt kontextusra. \par

Ez a döntés növelte a 
\begin{listing}
	\item láthatóságot, egy megfigyelő (vagy \emph{monitoring}) rendszernek csupán egy kérés alapján meg tudja határozni a kérés teljes természetét
	\item megbízhatóságot, mivel a rendszer könnyebben helyreáll egy részleges hiba után
	\item skálázhatóságot, mivel azzal, hogy nem tárol a szerver kérések között semmilyen állapotot, gyorsabban fel tudja szabadítani használt erőforrásait, illetve nem kell menedzselnie az erőforrásokat akár több kérésen keresztül
\end{listing}

Ezzekkel szemben negatívumként említhetjük a romlott hálózati teljesítményt.

\subsubsection{Gyorsítótár}

Az előbb említett hátrányra válaszul vezessük be a modellünkbe gyprsítótárazhatóság fogalmát. Ez megköveteli hogy egy válaszban szereplő adat implicit vagy explicit módon meg legyen jelölve hogy gyorsítótárazható-e. Ha egy válasz gyorsítótárazható, akkor egy kliens oldali gyorsítótárnak módjában áll újra felhasználni azt az adatot későbbi, az eredeti kéréssel ekvivalens kérésekhez.

\subsubsection{Egységes interfész}

Azzal, hogy a REST szolgáltatások egy egységes interfészen kommunikálnka a klienssel, függetlenítjuk a klienset a szolgáltatás implementációjától. Annak érdekében, hogy egy interenet szintű REST szolgáltatásoknak egy egységes interfészt határozzunk meg, (azaz egy egyezményt kliens és a szolgáltatás között, ami definiálja kommunkációjuk formáját), ezt szabványok felhasználásával kell megtennünk.

\begin{listing}
	\item Erőforrás azonosítás: URI standard \cite{RFC3986}
	\item Erőforrás manipuláció: HTTP standard \cite{RFC2616}
	\item Önleíró üzenetek: MIME típusok \cite{RFC2045}
	\item HATEOAS: hyperlink-ek és URI template-ek \cite{RFC6570}
\end{listing}

\subsubsection{Réteg alapú rendszer}

Hierarchikus rétegeket alakítunk ki azáltal, hogy az adott rétegek csak a velük közvetlenül kapcsolatban lévő rétegekről tudnak, csak ezekkel tudnak kommunikálni. Azzal, hogy limitáljuk az egyes rétegek tudását a rendszerről, csökken az egész rendszer komplexitása. A rétegek alapú architektúra adta lehetőségeket használhatjuk legacy szolgáltatások, komponensek szeparálására, illetve megkönnyíti az ezekről való átállást.

\subsubsection{REST és HTTP}

A REST szolgáltatások HTTP kérések fogadása és válaszok küldése útján bonyolítják le a kliens és szerver közti kommunikációt. A HTTP kérések és válaszok szemantikai jelentését az \ref{tab:restAndHttp} ábrán láthatjuk.

\section{BCrypt}

A 90-es évek végén a mikroprocesszorok gyorsulása rohamosan növelte kibertámadások mögött álló számítási teljesítményt, míg számos autentikációs séma titkos, felhasználók által válaszott jelszavakon alapult, amelyek hossza és véletlenszerűsége közel konstans maradt. Niels Provos és David Mazières erre a problémára megoldásként egy jövőbiztos, a hardveres fejlődéssel lépést tartó jelszó sémát, és ennek részeként a BCrypt algoritmust hozta létre. \cite{provos1999future}. \par

Az algoritmust szándékosan lassúra és költségesre tervezték. Ez jó designbeli döntésnek bizonyult, mivel ezzel csökkenthető a \emph{brute-force}, azaz nyers számításierőn alapuló támadások hatékonysága. A BCrypt által nyújtott védelem tovább növelhető azzal, ha az algoritmust többször lefuttatjuk, minden új körben az előzőből kapott értéket véve hash-elendő értéknek. Ezzel a technikával még tovább inkrementálható a titkosítás hatékonysága. \par

A projektben először SHA-256 algoritmussal titkosítottuk a felhasználók jelszavát. Az emögött álló érvek a gyors, hatékony működés, illetve az alacsony erőforrásigény voltak. A későbbiekben viszont pont ezekből az okokból kifolyólag, illetve mivel hatékonysága jelentős teljesítménybeli növekedést élvez, ha GPU-n implementáljuk. Ez a tulajdonsága jelentősen növeli a brute-force alapú támadások hatékonyságát \cite{patra2021cryptography}. A BCrypt algoritmus nem rendelkezik ezzel a tulajdonsággal, nehéz hatékonyan implementálni GPU-n. Azzal, hogy egy eleve lassabb, nem párhuzamosítható algoritmust használunk, amit könnyen tudunk skálázni a több körös titkosítással, nem rontjuk sem az alkalmazást rendeltetésszerűen használó felhasználók élményét, sem az alkalmazás hatékonyságát. Amennyiben egy jogosult személy kívánja magát autentikálni, az algoritmus futási ideje szinte elhanyagolható lesz. \par


\subsection{Web Push API}

web push api