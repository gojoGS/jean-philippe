\section{A program struktúrája, bemutatása}

\section{Tervezési minták}

\subsection{Factory}

factory

\subsection{Builder}

builder

\subsection{Adapter}

adapter

\subsection{Strategy}

A \emph{Strategy} viselkedési minta lehetővé teszi, hogy definiálva egy általános algoritmus interfészét, algoritmusok egész családját vagyunk képesek létrehozni, melyek mindegyike egy lehetséges, érvényes implementációját enkapszulálja az algoritmusnak, és kölcsönösen felcserélhetővé teszi őket. A minta lehetővé teszi az algoritmus interfészének és valós implementációinak elválasztását, ami azt eredményezi, hogy ezek bármikor különbözhetnek kliensektől, amelyek felhasználják, anélkül, hogy a külvilág számára észlelhető viselkedésük inkonzisztens lenne. \par



\section{Használt könyvtárak és keretrendszerek}

\subsection{Lombok}

A Lombok könyvtár, aminek célja a repetatív kódrészletek (úgynevezett boilerplate kód) írásának elkerülése, a fejlesztői élmény javítása. A legtöbb esetben nem terjeszti ki a Java funkcióit, hanem már meglévő funkciók használatát teszi kényelmesebbé. \par

Bizonyos keretrendszerek disziplínái megkövetelik például a getter-ek és setter-ek, bizonyos contructor-ok definiálását. Ilyen esetben hasznlhatjuk a Lombok, ami a build folyamatunkba beépülve, valid Java bytecode-ot generál automatikusan azokban az osztályokban, ahol bevezettük az annotációit.\par

Amennyiben kíváncsiak vagyunk arra, hogy a Lombok milyen transzformációkat hajt végre a kódunkon, vagy egyszerűen csak meg akarunk válni tőle, és eltávolítani a függőségink közül, a kódbázisunkat pedig megtisztítani a Lombok annotációktól, abban az esetben erre is van lehetőség. \par

A delombok nevű eszköz előállítja azokat az osztályok forráskódját, amikben Lombok annotációt használtunk, eltávolítva az annotációkat és helyükre a velük ekvivalens kód kerül, amelyet eddig a Lombok generált. \par


\subsubsection{@NoArgsConstructor, @AllArgsConstructor}

A \lstinline|@NoArgsConstructor|-t egy osztályra helyezve egy argomentumok nélküli konstruktort fog generálni. Amennyiben ez nem lehetséges, például egy \lstinline|final| mező miatt, a generálási folyamat egy fordítási hibát fog eredményezni \cite{lombokConstructorDocumentation}. Ez megkerülhető úgy, ha az annotáció \lstinline|force| paraméterének \lstinline|true| értéket adunk meg. Ezzel elérjük, hogy a \lstinline|final| mezők is inicializálva lesznek \lstinline|0|, \lstinline|false| vagy \lstinline|null| értékkel, azonban olyan mezők esetében, a \lstinline|@NonNull| annotációval végzünk null vizsgálatot, ez a vizsgálat nem fog legenerálódni, megtörténni. \par

\snippet{NoArgsConstructorBefore}{@NoArgsConstructor}
\snippet{NoArgsConstructorAfter}{@NoArgsConstructor delombok után}

A project elkészítése során Spring Data JPA entitás osztályokat láttunk el \lstinline|@NoArgsConstructor| annotációval, mivel a JPA megköveteli egy ilyen konstruktor létezését. \par


A \lstinline|@AllArgsConstructor| egy konstruktort generál egy osztálynak, annak egy-egy mezőjéhez tartozó egy paraméterrel. Amennyiben egy mező el van látva a \lstinline|@NonNull| annotációval, a generált konstruktor egy null check-et fog végezni azon a mezőn.

\snippet{AllArgsConstructorBefore}{@AllArgsConstructor}
\snippet{AllArgsConstructorAfter}{@AllArgsConstructor delombok után}

Az \code{@AllArgsConstructor} hátránya, hogy csak azok a mezők szerepelnek a konstruktor paraméterei között, amelyeket az osztályban deklaráltunk, azaz a szülő osztály mezőihez tartozó argomentumok nem szerepelnek a gyermek osztály generált konstruktorában. Ennek ellenére használata eredményes volt a projectben, például DTO-k és adattagokkal rendelkező enumok definniálásánál.


\subsubsection{@Getter, @Setter}

Egy osztály bármilyen mezőjét annotálhatjuk a  \lstinline|@Getter|-rel vagy \lstinline|@Setter|-rel, a Lombok automatikusan generálni fog egy alapértelmezett getter/setter metódust. \par

A \code{foo} mező alapértelmezett getter-je a \code{getFoo} nevű metódus, illetve \code{boolean} típusú mező esetében \code{isFoo}, ami a \code{foo} mezőt téríti vissza. Ezen mező alapértelmezett setter-je egy \code{setFoo} nevű, egy paraméteres metódus, amely paraméter típusa azonos a mező típusával. \par

A generált metódus publikus láthatóságú lesz, amennyiben ezt nem írjuk felül az annotációban elhelyezett \code{AccessLevel} értékkel. Ennek lehetségek értékei \code{PUBLIC}, \code{PROTECTED}, \code{PACKAGE} és \code{PRIVATE}, melyek rendre a Java nyelv láthatóságainak felelelnek meg. \par

Ezeket az annotációkat nemcsak mezőkön, hanem osztályokon is alkalmazhatjuk. Ebben az esetben az annotációhatása ekvivalens azzal, h az osztály minden, nem statikus mezőjét annotáltuk volna. Kivételt képeznek azok a mezők, amelyeket menuálisan annotálunk és láthatóságnak \code{AccessLevel.NONE}-t határozunk meg. \par

\snippet{GetterSetterBefore}{@Getter, @Setter}
\snippet{GetterSetterAfter}{@Getter, @Setter delombok után}

\subsubsection{@ToString}

Bármely osztály annotálható \code{@ToString}-el, amely felülírja az osztály \code{toString} metódusát, egy, a Lombok által generált implementációval. Alapértelmezetten, ezen implementáció által visszaadott string tartalmazza az osztály nevét, követve a az osztály nem statikus mezőinek nevével és ezek értékeivel, a deklarálásuk sorrendjében. \par

Amenability nem akarjuk, hogy minden mező megjelenjen a metódus output-jában, jelezhetjük a nem kívánt mezőket a \code{@ToString.Exclude} annotációval, felsorolhatjuk a tartalmazni kívánt mezők nevét a \code{@ToString} metódus \code{includeFieldNames} paraméterében, vagy a \code{onlyExplicitlyIncluded} paraméternek megadott \code{true} értékkel és a tartalmazni kívánt mezőkön elhelyezett \code{@ToString.Include} annotációval. Amennyiben a \code{callSuper} praméternek \code{true} értéket adunk, a visszaadott string tartalmazni fogja a szülő osztály \code{toString} metódusának output-ját \par

\subsubsection{@Builder}

A \code{@Builder} annotációval annotált \code{Foo} osztályhoz, konstruktorhoz vagy metódushoz (a későbbiekben ez \emph{target}) generálódik egy belső, statikus, \code{FooBuilder} nevű osztály (a későbbiekben ez \emph{builder}). \par

A builder osztály tartalmaz target egy-egy paraméteréhez vagy mezőjéhez tartozó, privát, nem statikus, nem \code{final} mezőt, egy package private no-args konstruktort. Builder minden mezője rendelkezik egy setter-szerű metódussal, ami a mező értékéhez a paraméterként kapott értéket rendeli, és a builder példányt adja vissza, ezzel elhetővé téve a metódushívások egymásba láncolását. Builder rendelkezik egy \code{build} metódussal, amely meghívásakor meghívódik target (amennyiben target osztály, annak egy megfelelő konstruktora), builder mezőinek értékével, és ugyanazt a típust adja vlssza, mint target (osztály esetében ez target). \par

A target-et tartalmazó osztályban (amennyiben target osztály, ez target)  generálódik egy \code{builder} metódus, ami builder egy új példányát hozza létre. \par

\snippet{BuilderBefore}{@Builder}
\snippet{BuilderAfter}{@Builder delombok után}

\subsubsection{@Slf4j}

A \code{@Log} annotáció számos variánssal rendelkezik. Ezen variánsok egy-egy log kezelő megoldáshoz készülnek, mivel számos ilyen megoldás van a JVM platformon, egy annotáció családról beszéhetünk. Közös tulajdonsága a család tagjainak, hogy az annotáció elhelyezése után elérhető egy \code{log} nevű objektum, ami az adott logolási megoldás logger példánya. Az \code{@Slf4j} az azonos nevű loggert teszi elérhetővé. \par

A fejlesztés során hasznosnak bizonyult, mivel a kódbázis tisztább és kisebb lett, a kód kevesebb zajt tartalmaz.

\snippet{Slf4jBefore}{@Slf4j}
\snippet{Slf4jAfter}{@Slf4j delombok után}





\subsection{Spring Boot}

boot

A Spring Boot Starter Data JPA egyszerűbbé teszi az adatbázisokkal való kommunikációt, egyszerűbbé teszi az adatelérési réteg kialakítását, mindezt oly módon, hogy a megoldás átlátható, könnyen búvíthető legyen. Amennyiben úgy döntünk, hogy nem kívánunk mi magunk adatbázis sémákat, entitásokat, lekéréseket létrehozni, hanem ezt a feladatot a Spring Data JPA-nak delegáljuk, lehetőségünk van arra, hogy a JPA hozza létre az adattáblákat, lekérdezéseket és egyéb SQL parancsokat, és ezek eredményét kezelje.

\subsubsection{Perzisztens entitások}

Ahhoz, hogy perzisztálhassuk egy osztály egy példányát egy adatbázisban 

\begin{listing}
	\item annotálnunk kell a \code{@Entity} annotációval
	\item rendelkeznie kell egy argomentumok nélküli konstruktorral
	\item minden perzisztálni kívánt mezőjének rendelkeznie kell publikus getter-el és setter-el
	\item egy mezőt meg kell jelölnünk az \code{@Id} annotációval, ami az adatbázisbeli azonosítója lesz
\end{listing}

Opcionálisan annotálhatjuk az entitás osztályt a \code{@Table} annotációval, amelynek \code{name} paramétereként megadott string lesz az entitásokat tartalmazó tábla neve. \par

Az entitás egy-egy mezőjéhez tartozó oszlop neve a mező nevével ekvivalens, amennyiben ezt nem írjuk felül a mezőn elhelyezett \code{@Column} annotáció \code{name} paraméterének adott string-gel. \par

Ügyeljünk arra, hogy semmiképen sem használjunk SQL kulcsszavakat táblák vagy mezők neveként.

\subsubsection{@Converter, AttributeConverter}

Amennyiben egy olyan értéket akarunk perzisztálni, ami nem

\begin{listing}
	\item JPA entitás
	\item JPA entitások kollekciója
	\item primitív Java típusok és ezek wrapper osztályai, illetve \code{String}
\end{listing}

\noindent biztosítanunk kell egy osztályt, ami definiál egy kölcsönös leképezést egy perzisztálható és a jelenlegi perzisztálni kívánt típus között. \par

Ennek eléréséhez definiálnunk kell egy osztályt, ami annotálva van a \code{@Converter} annotációval, illetve implementálja a \code{AttributeConverter<A, C>} generikus interfészt, ahol A az az entitás attribútum típus, amiből kiindulunk, és perzisztálni kívánjuk, C pedig az, amit valóban el tud tárolni az adatbázis. \par

Az \code{AttributeConverter<S, T>} interfésznek két definiált metódusa van:

\begin{listing}
	\item \code{C convertToDatabaseColumn(A attribute)}, ez konvertálja az entitásból kapott attribútumot egy, adatbázis által tárolható értékké
	\item \code{A convertToEntityAttribute(C column)}, ez konvertálja az adatbázis egy oszlopában szereplő értéket entitás attribútummá.
\end{listing}

Az interfészt implementáló osztályt annotálnunk kell a \code{@Converter} annotációval annak érdekében, hogy alkalmazható legyen az átváltás. Ammenyiben az annotáció \code{autoApply} paraméterének igaz értéket adunk, a perzisztancia ellátójának (angolul \emph{persistence provider}) muszály automatikusan alkalmazni a konvertert minden entitás minden perzisztált attribútumára, amire alkalmazható, kivéve, ahol az attribútumon elhelyezett \code{@Convert} annotáció ezt felülírja \cite{converterDocumentation}.

A project elkészítése során \code{enum} értékek konvertálására használtuk.

\snippet{DishType}{DishType}
\snippet{DishTypeConverter}{DishTypeConverter}

\subsubsection{JPA kapcsolatok}

A JPA kapcsolatoknak két fajtája van:
\begin{listing}
	\item egyirányú (\emph{unidirectional})
	\item kétirányú (\emph{bidirectional})
\end{listing}

Két entitás közötti kapcsolat definiálásának nincs hatása az entitások adatbázisra való leképezésének, csupán azt definiálja, hogy milyen irányban tudjuk használni a kapcsolatot a domain modellünkben. \par

Objektumok esetében a kétirányú kapcsolat egy definiált fogalom, szemantikája jól meghatározható, a relációs adatbázisok esetében azonban ez a fogalom nem létezik, csupán egyirányú kapcsolatok vannak (foreign key-ek formájában). Így a Hibernate, ami az objektumok relációkra való leképzést végzi, két egyirányú kapcsolattal modellezi a kétirányú kapcsolatot. A Hibernate a kapcsolat mindkét oldalának változásait követi, így például ha a kapcsolat pontosan egy résztvevője megváltozik, az az adatbázisban is meg fog változni, de ez a változás nem fog manifesztálódni a kapcsolat másik oldalán, azaz inkonzisztencia lép fel az adatbázisban. Ezt a problémát oldja meg a birtokló oldal, birtoklott oldal meghatározása. \par

Ha Hibernate csupán a birtokló oldal változásait követi nyomon, így ha a birtokló oldalon bármilyen változás történik a kapcsolatban, azt képes leképezni a kapcsolat másik felére, a kapcsolat birtokolt oldalán történő változtatásokat pedig nem követi. Ez a megoldás megoldja a kétirányú kapcsolatokból eredő inkonzisztenciákat. A birtokolt oldal könnyen felismerhető onnan, hogy értéket adunk a kapcsolatot jelző annotáció \code{mappedBy} paraméterének. \par

\subsubsection{@OneToOne}

Amennyiben egy mezőt a \code{@OneToOne} annotációval látunk el, egy egy-az-egyhez leképezést tudunk létrehozni két entitás között, azaz egy olyan leképezést, amelyben egy entitás egy példányához egy másik entitás legfeljebb egy példányát rendeljük. Az egy-az-egyhez leképezést több módon implementálhatjuk:

\begin{listing}
	\item \emph{foreign key} használatával
	\item \emph{közös primary key} használatával
	\item \emph{join table} használatával
\end{listing}

A project elkészítése során foreign key használatával implementáltuk a kapcsolatot. Ebben az implementációban a \code{@OneToOne}-nal annotált mezőt a \code{@JoinColumn} annotációval is el kell látni, amelynek \code{name} paraméterével adjuk meg a foreign key-t tartalmazó oszlop nevét. Ez az oszlop a kapcsolatban szereplő birtokló fél táblájában fog szerepelni. \par

Amennyiben a kapcsolat egyirányú, a birtokló fél egyértelműen meghatározható: az az entitás, amely a \code{@OneToOne}-nal annotált mezőt tartalmazza.\par

Ha a kapcsolatot kétirányúvá szeretnénk bővíteni, a egy-az-egyhez kapcsolat másik felét képző entitásban el kell látnunk a kapcsolat másik felének mezőjét a \code{@OneToOne} annotációval, amelynek \code{mappedBy} paraméterével adjuk meg, hogy a másik oldal melyik mezője hivatkozik rá. \par

\snippet{OneToOne}{RestaurantTable mint birtokló, EndUser, mint birtokolt}

\subsubsection{@OneToMany, @ManyToOne}
A \code{@OneToMany} és a \code{@ManyToOne} annotációkkal egy-a-többhöz és több-az-egyhez kapcsolatokat tudunk definiálni. Amennyiben külön-külön alkalmazzuk őket, ezek egyirányú kapcsolatokat határoznak meg, azonban ha egymással párban, két entitáson, akkor egy darab, kétirányú kapcsolatot definiál az entitások között. \par

A projectben ezzel a két annotációval reprezentáltuk a rendelések és a pincérek közötti kapcsolatot, azaz egy pincérnek több, hozzá rendelt rendelése lehet, míg egy rendeléshez legfeljebb egy pincér rendelhető.

\snippet{OneToMany}{}

\subsubsection{@ManyToMany}

A project során a rendelések és a rendelésekben szereplő termékek közötti kapcsolatot több-a-többhoz kapcsolattal írtuk le, mivel egy rendeléshez több termék tartozik, illetve egy termék több rendeléshez is hozzárendelhető. \par

Több féle módon implementálhatjuk: 

\begin{listing}
	\item \emph{join table} használatával
	\item egy új, közvetítő entitás létrehozásával, ami enkapszulálja a kapcsolatban résztvevő feleket (például egy ItemsOfOrder entitás, ami tartalmazná hogy melyik rendelés mely termékkel van asszociálva)
	\item \emph{composite key} használatával
\end{listing}

A project során \emph{join table}-t felhasználva implementáltuk. A birtokló oldalon definiálni kell a kapcsolatokat tartalmazó tábla nevét, illetve hogy a kapcsolatban résztvevő felek id-jai ennek a táblának mely oszlopaiban szerepelnek \par

\snippet{ManyToMany}{Order, mint birtokló, Item, mint birtokolt}

\subsubsection{JpaRepository, @Repository}

A Spring Data JPA által biztosított \code{JpaRepository} egyszerű, könnyen használható és kibővíthető megoldást nyújt az adatelérési réteg létrehozására. Csupán egy interface-t kell létrehoznunk, ami kiterjeszti a \code{JpaRepository<T, ID>} generikus interfészt, ahol a T a perzisztált entitás típusa, ID pedig ezen entitás azonosító mezőjének típusa \cite{jpaRepositoryDocumentation}. A \code{JpaRepository} által definiált metódusok lehetővé teszik az alapvető  DUCS adatbázis műveletek (azaz Delete, Update, Create, Select) használatát perzisztált entitásokra. \par

Amennyiben az alap műveleteken túlmutató lekérdezéseket szeretnénk definiálni, erre is van módunk. A \code{JpaRepository}-t kiterjesztő interfészben van módunk további metódusokat deklarálni, amelyek, ha nevük bizonyos szemantikai szabályokat követnek, interpretálva lesznek, mint SQL query-k. Az említett szemantikai szabályok az \ref{tab:JpaRepository} táblázatban találhatóak. \par

A \code{@Repository} annotációval jelezzük a Spring keretrendszer felé, hogy az interfész egy \emph{Repository}, azaz egy mechanizmus entitások tárolásra, kinyerésésre és keresésésre. A Spring 2.5-ös verzója óta a \code{@Component} annotáció egy specializációjaként is szolgál, azaz imlpementációi automatikusan detektálódnak.



\subsection{Vaadin}

vaadin

\section{Felhasznált szabványok, ajánlások}

\section{UUID}

Az univerzálisan egyedi azonosító (angolul \emph{Universally unique identifier}) vagy UUID egy ITF szabvány, amit az RFC 4122 definiált. A fő motiváció használatára az, hogy nem szükséges egy központi szerv bevonása a létrejövő azonosítók adminisztrálására, így generálásuk teljes mértékben automatizálható. A UUID-k fix mérete miatt, ami 128 bit, jelentősen kisebb, mint a legtöbb alternatív megoldás. Kompakt méretéből következik, hogy használata optimálisabb teljesítményt eredményez rendező és hasító algoritmusok, illetve adatbázisban való tárolás esetében. Az említett RFC-ben leírt generáló algoritmus akár másodpercenként 10 millió allokációt tud elvégezni gépenként, melyből kifolyólag a UUID-t tranzakciók azonosítójaként is használhatjuk. \par

A UUID azonosítók valójában nem \emph{teljesen} egyediek, azaz van esély arra, hogy két generált azonosító ugyanazzal az értékkel fog rendelkezni, ütközni fognak. Azonban ennek a valószínűsége elenyésző. A UUID ütközést tekinthetjük a születésnap probléma egy speciális esetének. Annak az esélye, hogy egy populációban, egymástól függetlenül kiosztott $x$ azonosító közül $n$-et kiválasztva $p$ valószínűséggel legyen köztük legalább kettő egyező az

\begin{equation*}
	n = 0.5 + \sqrt{0.25 - 2\times(\ln q)\times x}
\end{equation*}

formulával kiválóan lehet közelíteni \cite{mathis1991generalized}, ahol $q = 1 - p$. Ebből adódóan ahhoz, hogy legalább 50\%-os eséllyel generálódjon legalább két UUID,

\begin{equation*}
	n \approx 0.5 + \sqrt{0.25 + 2\times(\ln 2)\times2^{122}} \approx 2.71 \times 10^{18}
\end{equation*}

azonosítót kellene generálnunk. Ehhez, a másodpercenkénti 10 milló generált azonosítóból kiindulva megközelítóleg 8587 év szükséges, azaz ennyi idő szükséges ahhoz, hogy 50\%-os valószínűséggel előidézzünk egy UUID ütközést. \par

Egy étterem egy, a rendszerbe felvett asztalához tartozó, automatikusan létrehozott végfelhasználói fiók "felhasználóneve" egy UUID, amit az említett fiók létrehozásakor generálunk. Ez egy optimális megoldás, hiszen az étteremnek nem kell "kitalálni" valamilyen egyedi azonosítót a végfelhasználói fióknak, így gyorsabb és felhasználóbarátabb az asztalok a rendszerbe való felvezetésének folyamata. A UUID tulajdonságaiból kiindulva ezek az azonosítók ésszerű keretek között valóban egyediek lesznek.

\subsection{REST API}

Az állapotreprezentáció-transzfer (angolul \emph{representational state transfer}) vagy REST nem egy konkrét, jól definiált standard, sokkal inkább egy architektúrális stílus, ami elterjedt technológiákat és szabványokat használ fel web alapú szolgáltatások tervezésére és implementálására \cite{richards2006representational}. \par

\emph{Roy Thomas Fielding} doktori disszertációjában számos megkötést tett arra, hogy hogyan definiálható egy REST szolgáltatás architektúrája \cite{fielding2000architectural}. 

\subsubsection{Kliens-szerver architektúra}
Emögött a \emph{separation-of-concerns} elv áll. Azzal, hogy elválasztjuk a felhasználói interfészt az adattárolás szerepkörétől, skálázhatóbbá és szélesebb körben portolhatóvá válik a felhasználói interfész. Emellett ez az elkülönülés megengedi, hogy külön a szerverkomponensek egymástól elkülönülve fejlődjenek.

\subsubsection{Állapotmentes}
A kliens és szerver közti kommunikációnak állapotmentesnek kell lennie, oly módon, hogy a klienstől érkező minden kérése tartalmazza az összes, a kérés feldolgozásához szükséges információt, nem hagyatkozhat bármilyen, a szerveren tárolt kontextusra. \par

Ez a döntés növelte a 
\begin{listing}
	\item láthatóságot, egy megfigyelő (vagy \emph{monitoring}) rendszernek csupán egy kérés alapján meg tudja határozni a kérés teljes természetét
	\item megbízhatóságot, mivel a rendszer könnyebben helyreáll egy részleges hiba után
	\item skálázhatóságot, mivel azzal, hogy nem tárol a szerver kérések között semmilyen állapotot, gyorsabban fel tudja szabadítani használt erőforrásait, illetve nem kell menedzselnie az erőforrásokat akár több kérésen keresztül
\end{listing}

Ezzekkel szemben negatívumként említhetjük a romlott hálózati teljesítményt.

\subsubsection{Gyorsítótár}

Az előbb említett hátrányra válaszul vezessük be a modellünkbe gyorsítótárazhatóság fogalmát. Ez megköveteli hogy egy válaszban szereplő adat implicit vagy explicit módon meg legyen jelölve hogy gyorsítótárazható-e. Ha egy válasz gyorsítótárazható, akkor egy kliens oldali gyorsítótárnak módjában áll újra felhasználni azt az adatot későbbi, az eredeti kéréssel ekvivalens kérésekhez.

\subsubsection{Egységes interfész}

Azzal, hogy a REST szolgáltatások egy egységes interfészen kommunikálnak a klienssel, függetlenítjük a klienset a szolgáltatás implementációjától. Annak érdekében, hogy egy internet szintű REST szolgáltatásoknak egy egységes interfészt határozzunk meg, (azaz egy egyezményt kliens és a szolgáltatás között, ami definiálja kommunkációjuk formáját), ezt szabványok felhasználásával kell megtennünk.

\begin{listing}
	\item Erőforrás azonosítás: URI standard \cite{RFC3986}
	\item Erőforrás manipuláció: HTTP standard \cite{RFC2616}
	\item Önleíró üzenetek: MIME típusok \cite{RFC2045}
	\item HATEOAS: hyperlinkek és URI template-ek \cite{RFC6570}
\end{listing}

\subsubsection{Réteg alapú rendszer}

Hierarchikus rétegeket alakítunk ki azáltal, hogy az adott rétegek csak a velük közvetlenül kapcsolatban lévő rétegekről tudnak, csak ezekkel tudnak kommunikálni. Azzal, hogy limitáljuk az egyes rétegek tudását a rendszerről, csökken az egész rendszer komplexitása. A réteg alapú architektúra adta lehetőségeket használhatjuk legacy szolgáltatások, komponensek szeparálására, illetve megkönnyíti az ezekről való átállást.

\subsubsection{REST és HTTP}

A REST szolgáltatások HTTP kérések fogadása és válaszok küldése útján bonyolítják le a kliens és szerver közti kommunikációt. A HTTP kérések és válaszok szemantikai jelentését az \ref{tab:restAndHttp} ábrán láthatjuk.

\subsection{BCrypt}

A 90-es évek végén a mikroprocesszorok gyorsulása rohamosan növelte kibertámadások mögött álló számítási teljesítményt, míg számos autentikációs séma titkos, felhasználók által válaszott jelszavakon alapult, amelyek hossza és véletkenszerűsége közel konstans maradt. Niels Provos és David Mazières erre a problémára megoldásként egy jövőbiztos, a hardveres fejlődéssel lépést tartó jelszó sémát, és ennek részeként a BCrypt algoritmust \cite{provos1999future}. \par


\subsection{Web Push API}

web push api