A web egy gyorsan és dinamikusan fejlődő platform. A hagyományos asztali alkalmazások és natív mobilalkalmazások helyét folyamatosan veszik át a webapplikációk, illetve évről évre több teret nyernek a progresszív webalkalmazások. Számos de jure és de facto szabvány biztosítja rendeltetés szerű működését, melyeket az (TODO IETF vagy mi a rák volt a neve na azt ide) felügyel, fejleszt. \newline

Az utóbbi tíz év alatt rohamos fejlődésen esett át a web. Ezen évtized alatt megnövekedett az igény weblapok, -alkalmazások és -szolgáltatások fejlesztésére, és ezzel együtt bővült a fejlesztők számára elérhető eszközök kínálata. Mint frontend, mint backend oldalon számos kiforrott, iparban használt és aktív közösséggel rendelkező megoldást találhatunk igényeink kielégítésére, és az opciók folyamatosan bővülnek. \newline

A web nemcsak fejlesztői szempontból növekszik új, hasznos funkciókkal, hanem végfelhasználóként is egyre több hasznos funkció érhető el. Ezek egy része szabványokból ered, míg másrészt ipari és fejlesztőközösségi trendek befolyásolják a végfelhaszbálói élményt. A Mozilla \newline

Számos a natív szoftverek fejlesztéséről áttérünk a webes megoldásokra \newline

A Java, illetve a JVM platform kiváló alapot nyújt stabil, ipari minőségű megoldások fejlesztésére. A Java nyelv fejlődése konzervatívnak és lassúnak tűnhet olyan ökoszisztémákhoz képest, mint a C$\sharp$ (TODO kijavítnai a sharpot)  és a CLR platform. \newline

A JVM ökoszisztéma azonban egy kiforrott alap, számos szabad és nyílt forrású megoldással, mint például a Spring, a Google Guava, az Apache Commons, és ezen megoldások választéka folyamatosan bővül. Számos kutatási területben élen jár, gondolhatunk itt a garbage collection-nel kapcsolatos technológiákra, JIT fordításra, optimalizációra, natív kód kezelésre.
\newline

A JVM nemcsak a Java nyelvnek ad otthont, hanem nyelvek egész családjának. Ilyen például a Kotlin, ami a mobil- és webfejlesztésben nyert szerepet az elmúlt években, vagy a Scala, ami főleg backend oldalon szerzett népszerűséget. Az említett nyelvek gyorsan fejlődnek, számos új nyelvi funkcióval bővülnek, és ezzel egy időben profitálnak a célplatform fejlődéséből. Emellett jó és kiforrott az interoperálási lehetőség Java kóddal, így a már meglécő Java könyvtárakat ezekben a nyelvekben is tudjuk használni. \newline

Kiváló toolchain-nel rendelkezik, mint például az Apache Maven, a Gradle vagy az JetBrains IntelliJ. Ezek az eszközök ipari erősségű szoftverek, amelyek könnyen elérhetőek a fejlesztők számára, és segítik a fejlesztési folyamatot.