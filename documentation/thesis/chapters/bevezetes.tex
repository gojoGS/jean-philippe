A web egy gyorsan és dinamikusan fejlődő platform. A hagyományos asztali alkalmazások és natív mobilalkalmazások helyét folyamatosan veszik át a webapplikációk, illetve évről évre több teret nyernek a progresszív webalkalmazások. Számos önkéntes szabvány biztosítja rendeltetés szerű működését, melyeket például a W3C vagy az IETF felügyel, fejleszt. Emellett teret nyertek a de facto szabványok is, mint a JSON.\par

Az utóbbi tíz év alatt rohamos fejlődésen esett át a web. Ezen évtized alatt megnövekedett az igény weblapok, -alkalmazások és -szolgáltatások fejlesztésére, és ezzel együtt bővült a fejlesztők számára elérhető eszközök kínálata. Mint frontend, mint backend oldalon számos kiforrott, iparban használt és aktív közösséggel rendelkező megoldást találhatunk igényeink kielégítésére, és az opciók folyamatosan bővülnek. \par

A web nemcsak fejlesztői szempontból fejlődik. A végfelhasználók biztonságának és felhasználói élményének javítása egy konstans napirendi pont mind a szabványokat létrehozó testületek, mind a webfejlesztői közösség vezetői körében. Előbbire példa, hogy az új protokollokat, mint a QUIC, WebPush vagy WebRTC, már eleve a biztonságot és titkosítást szemelőtt tartva lettek tervezve, míg utóbbira példa lehet a Mozilla csoport törekvése a web biztonságosabbá és elérhetőbbé tételére \cite{mozillaVision}\par

A webes megoldásokra való átállással azonban új problémákba ütközhetünk. Az egyik ilyen a frontend és a backend közti kommunikáció. Amennyiben a backend oldalon megváltozik egy JSON séma, a frontend oldalán ezt a változást követnünk kell, másképp az alkalmazásunk nem fog rendeltetésszerűen működni. Erről a lépésről azonban hajlamosak lehetünk megfeledkezni, vagy nagyobb szabásű projectek esetében megeshet, hogy nem is értesülünk róla. \par 

Mivel lesz frontend és backend kódbázisunk, növekedni fog az alkalmazásunk mentális modelljének és valós implementációjának a komplexitása. Számos olyan forráskód szintű entitás jön létre egyszerre a két kódbázisban, melyek logikailag össze vannak kötve, vagy akár azonosak, mégsem tudjuk egyszerre megváltoztatni a kettőt. A frontend és a backend szoftver kód szinten el van szeparálva, logikai szinten azonban függenek egymástól, és az inkonzisztenciákat nincs módunk észrevenni időben.\par

Ezekre a problémákra egy lehetséges megoldás az, hogy egyetlen, egységes kódbázisban létezik a frontend és a backend, így a frontend és a backend ugyanazokat szoftveres entitásokat használja, ami eddig két, elszeparált kódbázisban létezett.\par

A Java, illetve a JVM platform kiváló alapot nyújt stabil, ipari minőségű megoldások fejlesztésére. A Java nyelv fejlődése konzervatívnak és lassúnak tűnhet olyan ökoszisztémákhoz képest, mint a C\verb|#| és a CLR platform. \par

A JVM ökoszisztéma azonban egy kiforrott alap, számos szabad és nyílt forrású megoldással, mint például a Spring, a Google Guava, az Apache Commons, és ezen megoldások választéka folyamatosan bővül. Számos kutatási területben élen jár, gondolhatunk itt a garbage collection-nel kapcsolatos technológiákra, JIT fordításra, optimalizációra, natív kód kezelésre.
\par

A JVM nemcsak a Java nyelvnek ad otthont, hanem nyelvek egész családjának. Ilyen például a Kotlin, ami a mobil- és webfejlesztésben nyert szerepet az elmúlt években, vagy a Scala, ami főleg backend oldalon szerzett népszerűséget. Az említett nyelvek gyorsan fejlődnek, számos új nyelvi funkcióval bővülnek, és ezzel egy időben profitálnak a célplatform fejlődéséből. Emellett jó és kiforrott az interoperálási lehetőség Java kóddal, így a már meglévő Java könyvtárakat ezekben a nyelvekben is tudjuk használni. \par

Kiváló toolchain-nel rendelkezik, mint például az Apache Maven, a Gradle vagy az JetBrains IntelliJ. Ezek az eszközök ipari erősségű szoftverek, amelyek könnyen elérhetőek a fejlesztők számára, és segítik a fejlesztési folyamatot. \par

Azért választottam ezt a témát, mert a web egy globálisan elérhető, innovatív platform, ami lehetőséget nyújt érdekes, új technológiák kipróbálására, mint például a Java Spring Boot és Vaadin Flow keretrendszerek, amelyek ötvözésével áthidalható az fentebb taglalt "több kódbázis" probléma. A Java nyelv statikus, erősen típusos típusrendszere egy szigorú, de sok hibára figyelmeztető környezetet nyújt. Egy nagy kifejezőerővel rendelkező, modern, objektum orientált programozási nyelv, ami számos funkciót biztosít szoftvertervezési minták és komplex architektúrák felépítésére. \par

A szakdolgozat részeként egy rendelésfelvevő és -kezelő webalkalmazás jött létre éttermek részére, egy Java alapú megoldás formájában. \par

Az alkalmazás lefejlesztése módot ad a Java nyelvi elemeinek, szoftverfejlesztésben való felhasználhatóságának megvizsgálására, szoftveres minták implementálására, illetve egy egységes, Java kódbázis létrehozására. \par