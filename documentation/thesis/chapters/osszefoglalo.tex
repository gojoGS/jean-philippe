A dolgozat során lefejlesztettünk egy webalkalmazást, ami kihasználja a modern webtechnológiák nyújtotta lehetőségeket. Törekedtünk letisztult, zajmnetes felhasználói interfészt létrehozni, úgy, hogy ez ne menjem a vizuális élmény, interaktivitás és használhatóság rovására. \par

Az, hogy az alkalmazást nem asztali alkalmazásként, hanem webes platformra tervezett megoldásként hoztuk létre helyes döntésnek bizonyult. A központi backend lehetővé tette olyan funkciók megalkotását, melyek nem jöhettek volna létre egy lokálisan vagy decetnralizáltan futó backend esetében. Mivel a kliens oldalon csak a felhasználói felület megjelenítése történik, csupán egy alkalmas megjelenítő eszköz, illetve egy böngésző szükséges az alkalmazás használatához. Elkerültük a telepítés nehézkes folyamatait, továbbá a frissítések anélkül érkeznek meg a felhasználókhoz, hogy nekik ezzel kapcsolatban bármilyen teendőjük lenne. \par

A Java, illetve a JVM, mint szoftveres platform jó választásnak bizonyult. A fejlesztői közösség aktív és segítőkész, a köréjük kiépült ökoszisztéma számos könyvtárat és megoldást biztosított problémáinkra. Az általunk használt megoldások mindegyike szabad és nyílt forráskódú szoftver, amiket aktívan karbantart a közösség. Jól dokumentált, kiforrott keretrendszerekről és könyvtárakról van szó, melyek mindegyike jelentős ideje elérhető. A karbantartók rengeteg időt fordítottak stabilitásuk garantálására, és funnkcióik bővítésére. Számos könyv segítette használatukat, ezzel növelve a fejlesztés gyorsaságát. \par

A Spring keretrendszer kifejezetten hasznosnak bizonyult. Ez az ipari erősségű megoldás kiváló skálázható alkalmazások fejlesztésére. Nemcsak segíti, hanem jutalmazza is egy átgondolt, tiszta alkalmazásarchitektúra létrehozását, mindezt anélkül, hogy büntetné a fejlesztőt azért, ha mégis el szeretni térni az ajánlott idiómáktól. A Spring Security, illetve a Spring Boot Starter Data JPA modulok által biztosított funkciók lehetővé tették egy biztonságok, megbízható működésű, kiterjeszthető alkalmazás létrehozását. \par

A Vaadin szub-optimális választásnak bizonyult. Bár a mögötte álló fejlesztői csapat jelentős hangsúlyt fektet az innovációra, a keretrendszer nem rendelkezik kellő közösséggel. A keretrendszer weboldalán található fórumok inaktívak, jellemzőek a több éves, megoldatlan problémákról szóló szálak. A dokumentáció, bár nagyon hasznos azoknak, akik meg szeretnének ismerkedni a keretrendszerrel, azonban nélkülöznünk kell a komplexebb példákat. Sok esetben nem tér ki a sarkalatos esetekre, melyeknek feltárása egy hosszadalmas folyamatnak bizonyult. \par

A fejlesztési folyamatot magunk mögött tudva az alternatív frontend megoldások használata kecsegtetőbb. A legtöbb JavaScript frontend framework, legyen az akár React, Angular, VueJS vagy a Svelte, aktívabb közösséggel, jobb dokumentációval rendelkezik, számos könyv íródott alkalmazásukról. Bár használatukhoz rendelkeznünk kell HTML, CSS és JavaScript ismeretekkel az adott framework idiómái megértése mellett, ezek az alapok felhasználhatóak amikor esetleg áttérünk egyik megoldásról egy másikra. Továbbá az említett technilógiák szabadságot nyújtanak ötleteink megvalósítására, mindezt rugalmas módon. \par

A fejlesztés során nagy szerepet játszottak a fejlesztői eszközök. Az Apache Maven, mint build eszköz, ipari szinten elterjedt, szabad és nyílt forrású megoldás, ami nagyban segítette a project létrejöttét, bővülését és tesztelését. Kiválóan integrálható Spring-el, Lombok-al, illetve a Vaadin keretrendszerrel. A JetBrains IntelliJ kényelmes és hasznos fejlesztői környezetnek bizonyult. Kiváló Maven, Lombok, Spring és git integrációja, továbbá a refaktorálás folyamatát elősegítő funkcióinak hála lehetőségünk volt a fejlesztésre koncentrálni, az egyes eszközök közti kontextus váltás elkerülésével. \par

Az alkalmazás részeként létrehozott read-only REST API egységes, kibővíthető interfészt biztosít arra, hogy más rendszerek is tudjanak kommunikálni az alkalmazással. Mivel a REST egy elterjedt architektúrális megoldás, számos cikk, könyv és szoftveres komponens segítette kialakítását. Bár definíciója nem egzakt, irányelveit követve valóban skálázható és megbízható megoldásokat tudtunk fejleszteni. \par

A lefejlesztett alkalmazás kiváló precedenst nyújtott modern webes technológiák, szoftverfejlesztési eszközök és dizájn minták kipróbálására és alkalmazására a gyakorlatban. A dolgozat célját ebből a szempontból elérte. Ami a létrejött alkalmazást illeti, minden kitűzött funkciót sikerüt implementálnunk, méghozzá oly módon, hogy a végeredmény biztonságos, megbízható és bővíthető legyen.