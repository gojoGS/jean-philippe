Az elkészült alkalmazás bemutatásához (ami a \emph{Jean Philippe} nevet kapta) vegyünk egy életszerű példát. \par

Tegyük fel, hogy egy étterem üzemeltetője vagyunk, és nem vagyunk elégedettek a jelenlegi, hagyományos rendelési rendszerrel. Lassú, a pincérek hibázhatnak, nehéz kezelni és nyomonkövetni a rendeléseket, nem hatékony a kommunikáció a vendégekkel. Egy egyedi szofveres megoldás lefejlesztése túl drágának bizonyulna, nagy tőkebefektetéssel járna és túl nagy rizikót vállalnánk magunkra. Szükségünk lenne egy már meglévő szolgáltatásra. Egy hírdetés újtán rábukkanunk a Jean Philippe nevű szolgáltatásra. Weblapját felkeresve egy informatív, letisztult nyitóképernyő fogad. 

\image{Nyitóképernyő}{jp-home.png}

Az oldal funnkcióit böngészve rábukkanunk a már regisztrált, és a szolgáltatást aktívan használó éttermek listájára.

\image{Regisztrált éttermek}{jp-restaurants.png}

Mérlegeljük a szolgáltatás előnyeit

\begin{listing}
	\item minimális tőkebefektetés
	\item letisztult felület
	\item más éttermek is használják, így a vendégeknek már ismerős lehet a felhasználói felület
	\item szolgáltatás, tehát amennyiben úgy döntünk, hogy nem kívánjuk a továbbiakban használni, megválhatunk tőle
	\item web alapú, azaz platformfüggetlen, gyakoribbak a frissítések, nem szükséges hozzá setup folyamat
\end{listing}

és hátrányait

\begin{listing}
	\item az asztalokat fel kell szerelnünk valamilyen megjelenítő eszközzel
	\item az étlapot, asztalokat és a pincéreket fel kell vinni a rendszerbe
	\item a dolgozóinknak meg kell tanulni a felületet kezelni
	\item vendégeink egy részének személytelen lesz a rendelési folyamat
\end{listing}

Belátjuk, hogy az ellenérvek nemcsak ennél a megoldásnál jelentkeznének, így úgy döntünk, hogy regisztráljuk az éttermünket.

\image{Regisztráció}{jp-signup.png}

Választunk egy tetszőleges, még nem használatban lévő felhasználónevet, egy megfelelő jelszót. A \emph{Restaurant details} mezőit kitöltjük éttermünk nevével mottójával: \emph{Kis Gömböc - Ha jól akarsz lakni}. A regisztráció gombra kattintva egy értesítés közli velünk a sikeres regisztráció hírét, majd megjelenik a bejelentkező képernyő.

% TODO kurva gyorsan egy képet erről ide

Az autentikációhoz szükséges adatok megadása után megtörténik a bejelentkezés, az alkalmazás átirányít minket az éttermek felhasználói felületére. A bejelentkezés utáni első képernyőn éttermünk menüjét láthatjuk, ami jelenleg üres.

\image{Menü}{rs-menu.png}

Megkezdjük a papír alapú étlap felvitelét a rendszerbe. Az alkalmazás egységes felületet nyújt ét-, illetve itallap kialakítására. Pár elírás került egyes termékek leírásába, de ezeket a szerkesztés funkcióval sikerül korrigálni. Rájövünk, hogy egyes ételek szervírozása nem nyereséges már az étteremnek, ezért ezeket eltávolítjuk a törlés funkcióval.

\image{Étel létrehozás űrlap segítségével}{rs-add.png}

Itallapunk definiálása során tudjuk jelezni, hogy az egyes termékek tartalmaznak-e alkoholt, illetve, hogy diétás termékek-e.

\image{Ital létrehozása}{rs-add-beverage.png}

Ét- és itallapunk megadása után felmerül, hogy nem lenne-e esedékes az étterem mottójának megváltoztatása. A profil fülre navigálva látjuk, hogy módosítani tudjuk az éttermünk nevóét és leírását, új jelszót tudunk megadni, illetve ki tudunk jelentkezni. Több ötlet is érkezett az új leírásra, egyik sem bizonyult szubsztanciálisan jobbnak a jelenleginél, ezért visszaállítjuk az űrlapot eredeti állapotába a \emph{Reset} gombbal.

\image{Profil}{rs-profile.png}

Az onboarding folyamatot az asztalok felvételével folytatjuk. Megadjuk az egyes asztalko számát, amit eddig de facto standar-ként használtak dolgozóink egymás között, maximum kapacitásukat, illetve egy rövid leírást, hogy hol találhatóak.

\image{Asztalok}{rs-tables.png}
\image{Új asztal}{rs-new-table.png}

Az asztalokkal együtt végfelhasználók is létrejöttek. Miután felvettük a Jean Philippe-t fejlesztő csapattal a kapcsolatot, ők ajánlottak egy megjelenítési megoldást, amellyel optimális lesz a felhasználói élmény. A rendszer üzembe állítása után megkezdjük a végfelhasználók beléptetését. Az \emph{End user} fülre lépve láthatjuk az egyes asztalokhoz tartozó felhasználók azonosítóját. A \emph{New password} gombra kattintva megkapjuk az ezekhez tartozó jelszót is.

\image{Végfelhasználók}{rs-end-users.png}
\image{Új jelszó}{rs-new-password.png}

Végül vegyük fel a rendszerbe a pincéreinket.

\image{Személyzet}{rs-staff.png}
\image{Új pincér}{rs-add-staff.png}

Éttermünk sikeresen átállt a web alapú rendszerre, várjuk vendégeinket.

Egy személy úgy dönt, hogy szeretni elkölteni egy ebédet egy kellemes étteremben. Hallott a Jean Philippe keresési funkciójáról. Egy ideig böngészi az éttermek által szervírozott ételek listáját, rákeres bizonyos ételekre, többféle minimum és maximum ár kombinációval indít keresést. Végül választása a Kis Gömböc étteremre esik.

\image{Keresés}{jp-search.png}

Belépve az étterembe keres egy szabad asztalt, ahol egy kijelző fogadja egy letisztult kezdőképernyővel.

\image{Végfelhasználói kezdőképernyő}{eu-start.png}

Helyet foglal, majd megnyomja a \emph{Start session} gombot. Egy új képernyő fogadja, ahol ételeket tud hozzáadni a rendeléshez, illetve válogathat az itallapon szereplő opciók közül. Döntése végül 1 gulyáslevesre és 3 csirkepaprikásra esik. Az tételek kijelölése után megnyomja a \emph{Send order} gombot. Ezt követően átkerül egy úk képernyőre és várja az étterem válaszát.

\image{A rendelés}{eu-menu-ready.png}
\image{Várakozó képernyő}{eu-sent.png}

Az étterem, függetlenül hogy az alkalmazás mely pontján van, felugró értesítést kap az új rendelés beérkezéséről. A személyzet valamelyik tagja ezt észleli, és az \emph{Orders} fülre átlépve látja is, hogy az 1-es asztalnál új rendelés érkezett be. Megnyitva a rendelést látja a rendelés tartalmát, amit továbbít a konyha felé. A konyhai személyzet ad egy becslést a rendelés elkészítési idejére. A rendelést megnyitva hozzárendeli az egyik pincért, aki éppen nem annyira elfoglalt, és az egyeztetett becslést beírja az űrlap megfelelő mezőjébe, majd elmenti a változtatásokat. \par

\image{Új rendelés érkezett}{rs-new-order.png}
\image{Rendelések}{rs-orders.png}
\image{Rendeléssel kapcsolatos információk}{rs-estimate.png}

A vendég, asztalánál ülve változást vél felfedezni az eddigi képernyőn.

\image{Az étterem elfogadta a rendelést}{eu-updated.png}

Továbblépve látja rendelésének elkészítéséhez szükséges idő várható hosszát, illetve, hogy melyik pincér fogja tálalni. Várja az események további kibontakozását.

\image{Várakozás a rendelés elkészülésére}{eu-waiting.png}

Elkészült a rendelés, a kijelölt pincér átveszi a konyhától, a rendeléseket kezelő alkalmazott pedig lezárja a rendelést.

\image{Rendelés lezárása}{rs-close-order.png}

A vendég ismét változást vél felfedezni a felhasználói felületen. Egy új felületre navigált az alkalmazás, amely arról informálja, hogy rendelése készen van. Pillanatokon belül Egy pincér kihozza rendelését. A fogások elfogyasztása után úgy dönt, újabb rendelést ad le. A \emph{New order} gombra kattintva a már ismert rendelési felület fogadja. Megismétlődik ugyanez a rendelési folyamat. \par

\image{Kész a rendelés}{eu-order-ready.png}

Pár új rendelés után a vendég úgy dönt, hogy befejezné az étkezést, és fizetni szeretne. A \emph{Check out} opciót választva egy új felület tárul elé, ahol láthatja a végösszeget, és kiválaszthatja a fizetés módját. Egy ideig próbál választani a \emph{Készpénz}, \emph{Kártya} és \emph{Crypto} opciók között, végül a készpénzt választja. Fizetési szándéka jelezve lett az étterem felé, egy értesítést kaptak arról, hogy melyik asztal, milyen módon szeretne fizetni. Az asztalnál lévő felhasználói felület visszanavigált a session kezdő képernyőre, majd nem sokkal később egy pincér érkezett, akivel lebonyolítja a tranzakciót, majd elégedetten távozik az étteremből.

\image{Fizetési szándék jelzése}{eu-check-out.png}
\image{Az egyik asztal fizetni szeretne}{rs-ready-to-pay.png}
