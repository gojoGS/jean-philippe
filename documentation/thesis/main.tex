\documentclass[12pt]{report}
\usepackage[utf8]{inputenc}
\usepackage[T1]{fontenc}
\def\magyarOptions{chapterhead=unchanged}
\usepackage[magyar]{babel}
\usepackage{graphicx}
\graphicspath{{images/}}
\usepackage{times}
\usepackage{setspace}
\usepackage{biblatex} 
\addbibresource{refs.bib}
\usepackage{paracol}
\usepackage{parcolumns}
\usepackage{enumitem}
\usepackage{longtable}
\usepackage{amsmath}
\usepackage{multirow}
\usepackage{indentfirst}
\usepackage{color}

\setstretch{1.5}

\usepackage{titlesec}
\titleformat{\chapter}[hang]{\bf\large}{\thechapter}{14px}{}


\usepackage[a4paper,
left=3cm,
right=2cm,
top=3cm,
bottom=3cm]{geometry}

\usepackage{listings}
\usepackage{xcolor}

\definecolor{color-number}{HTML}{1D2021}
\definecolor{color-bg}{HTML}{f9f5d7}
\definecolor{color-comment}{HTML}{746f60}
\definecolor{color-keyword}{HTML}{fb4934}
\definecolor{color-string}{HTML}{FABD2F}
\definecolor{color-text}{HTML}{af3a03}
\definecolor{color-identifier}{HTML}{076678}

\lstdefinestyle{mystyle}{
	backgroundcolor=\color{white},   
	commentstyle=\color{color-comment},
	keywordstyle=\color{color-keyword},
	numberstyle=\color{color-number},
	stringstyle=\color{color-string},
	basicstyle=\ttfamily\footnotesize\color{color-text}\setstretch{1},
	identifierstyle=\color{color-identifier},
	otherkeywords={var},
	breakatwhitespace=false,         
	breaklines=true,                 
	captionpos=b,                    
	keepspaces=true,                	 
	numbers=left,                    
	numbersep=5pt,
	showspaces=false,                
	showstringspaces=false,
	showtabs=false,                 
	tabsize=4
}

\newcommand{\code}[1]{
	\lstinline{#1}
}
\newcommand{\snippet}[2]{
		\lstinputlisting[language=Java, caption={#2}, label={snippet:#1}]{code/#1.java}
}
\newcommand{\image}[2]{
	\begin{figure}[h]
		\caption{#1}
		\label{#2}
		\includegraphics[width=\linewidth]{figures/#2}
	\end{figure}
}

\newenvironment{listing}{
	\begin{itemize}[noitemsep,topsep=0pt]
}{
	\end{itemize}
}


\lstset{style=mystyle}
\renewcommand*{\lstlistingname}{Kódcsipet}

\renewcommand*{\contentsname}{Tartalomjegyzék}

\begin{document}
	\begin{titlepage}
	\begin{center}
		\vspace*{1cm}
		
		\textbf{\large{Szakdolgozat}}
		
		\vfill
		
		\includegraphics{ud-logo.png}
		
		\vfill
		
		\textbf{Bartha Zoltán}
		
		\vspace{0.8cm}
		
		Debrecen\\
		\today
		
	\end{center}
\end{titlepage}
	
	\begin{titlepage}
	\begin{center}
		\vspace*{1cm}
		
		\textbf{\large{Debreceni Egxetem \\ Informatikai Kar}}
		
		\vfill
		
		\textbf{Rendeléskezelő applikáció Java SE-nel}
		
		\vfill
		
		\begin{parcolumns}[colwidths={1=.4\linewidth}]{2}
			\colchunk{Tiba Attila \\ tanársegéd}
			\colchunk{Bartha Zoltán \\ programterevző informatikus}
		\end{parcolumns}
		
	\end{center}
\end{titlepage}
	
	\chapter*{Köszönetnyilvánítás}
	Köszönetnyilvánítás
	
	\tableofcontents
	
	\chapter{Bevezetés}
	
	A web egy gyorsan és dinamikusan fejlődő platform. A hagyományos asztali alkalmazások és natív mobilalkalmazások helyét folyamatosan veszik át a webapplikációk, illetve évről évre több teret nyernek a progresszív webalkalmazások. Számos de jure és de facto szabvány biztosítja rendeltetés szerű működését, melyeket az (TODO IETF vagy mi a rák volt a neve na azt ide) felügyel, fejleszt. \newline

Az utóbbi tíz év alatt rohamos fejlődésen esett át a web. Ezen évtized alatt megnövekedett az igény weblapok, -alkalmazások és -szolgáltatások fejlesztésére, és ezzel együtt bővült a fejlesztők számára elérhető eszközök kínálata. Mint frontend, mint backend oldalon számos kiforrott, iparban használt és aktív közösséggel rendelkező megoldást találhatunk igényeink kielégítésére, és az opciók folyamatosan bővülnek. \newline

A web nemcsak fejlesztői szempontból növekszik új, hasznos funkciókkal, hanem végfelhasználóként is egyre több hasznos funkció érhető el. Ezek egy része szabványokból ered, míg másrészt ipari és fejlesztőközösségi trendek befolyásolják a végfelhaszbálói élményt. A Mozilla \newline

Számos a natív szoftverek fejlesztéséről áttérünk a webes megoldásokra \newline

A Java, illetve a JVM platform kiváló alapot nyújt stabil, ipari minőségű megoldások fejlesztésére. A Java nyelv fejlődése konzervatívnak és lassúnak tűnhet olyan ökoszisztémákhoz képest, mint a C$\sharp$ (TODO kijavítnai a sharpot)  és a CLR platform. \newline

A JVM ökoszisztéma azonban egy kiforrott alap, számos szabad és nyílt forrású megoldással, mint például a Spring, a Google Guava, az Apache Commons, és ezen megoldások választéka folyamatosan bővül. Számos kutatási területben élen jár, gondolhatunk itt a garbage collection-nel kapcsolatos technológiákra, JIT fordításra, optimalizációra, natív kód kezelésre.
\newline

A JVM nemcsak a Java nyelvnek ad otthont, hanem nyelvek egész családjának. Ilyen például a Kotlin, ami a mobil- és webfejlesztésben nyert szerepet az elmúlt években, vagy a Scala, ami főleg backend oldalon szerzett népszerűséget. Az említett nyelvek gyorsan fejlődnek, számos új nyelvi funkcióval bővülnek, és ezzel egy időben profitálnak a célplatform fejlődéséből. Emellett jó és kiforrott az interoperálási lehetőség Java kóddal, így a már meglécő Java könyvtárakat ezekben a nyelvekben is tudjuk használni. \newline

Kiváló toolchain-nel rendelkezik, mint például az Apache Maven, a Gradle vagy az JetBrains IntelliJ. Ezek az eszközök ipari erősségű szoftverek, amelyek könnyen elérhetőek a fejlesztők számára, és segítik a fejlesztési folyamatot.
	
	\chapter{Tárgyalás}

	\section{A program struktúrája, bemutatása}

\section{Tervezési minták}

\section{Factory}

A \emph{single responsibility} elvéből kiindulva, minden osztálynak csak és kizárólag egy felelőséggel kell rendelkeznie \cite{martin2003agile}. Ezt figyelmebe véve el kell választanunk az objektumok példányosítását maguktól az objektumoktól. Ebben segít a \emph{Factory} minta. \par

A \emph{Factory} mintának több variánsa van. Létezik olyan verziója, amiben a létrehozandó osztály egy metódusán keresztül hozzuk létre az új objektumokat \cite{gamma1995elements}. A Spring keretrendszer bizonyos megkötései miatt egy ettől eltérő dizájnt használtunk. A Spring által nyújtott \code{@Autowired} annotációval tudjuk bizonyos objektumok létrehozásának szerepkörét átruházni a Springre. Ehhez azonban szükséges, hogy az osztály, amit példányosítani kívánunk, rendelkezzen egy argumentum nélküli konstruktorral. Vannak azonban olyan szolgáltatások, melyek működéséhez plusz információ, bizonyos paraméterek szükségesek. \par

Vegyük példaként a \code{RestaurantDetailsSetterService}-t.

\snippet{DetailsService}{RestaurantDetailsSetterService által definiált interfész}

Ez a szolgáltatás felelős az egyes éttermek tulajdonságainak megváltoztatásáért, mint például az étterem neve vagy leírása. A metódusoknak nem adjuk át paraméterként, hogy melyik étterem tulajdonságait kívánjuk megváltoztatni, hiszen így minden egyes hívásnál meg kell bizonyosodnunk róla, hogy a helyes azonosítót adjuk át. Ehelyett az implementálandó osztály felelőssége lesz ennek számontartása, egy privát mező formájában, amit egy, a konstruktorban paraméterként kapott értékkel inicializálunk. Bár ezt a megoldást is lehet helytelenül használni, kevesebbszer kell figyelmet fordítanunk a helyes használatra. Csupán egy helyen kell ügyelnünk a paraméter helyességére, továbbá, mivel ezen paraméter értéke nem változik, érdemesebb eltárolni a szolgáltatás létrehozásakor, mintsem minden metódushívásnál újra megadni. \par

Mivel a szolgáltatás konstruktora rendelkezik egy paraméterrel, nem tudjuk az \code{@Autowired} annotáció segítségével létrehozni. Viszont képesek vagyunk definiálni egy factory osztályt, aminek egy megfelelő metódusa felel az objektum létrehozásáért. \par

\snippet{Factory}{A factory interfésze}

A \code{get} metódusnak paraméterként megadunk minden kontextust a szolgáltatás létrehozásához. \par

\snippet{FactoryImpl}{Egy lehetséges factory implementáció}

A factory osztály a kontextus (azaz az étterem azonosítója) nyújtásán kívül elérést biztosít az adatelérési réteghez. A \code{RestaurantDetailsSetterService} implementációja privát belső osztályként szerepel a factory osztályban, ezzel növelve a factory osztály enkapszulációját. A factory minta ezen verziója a következő elemekből áll: \par

\begin{listing}
	\item egy létrehozandó szoftverkomponens interfésze (későbbiekben \emph{cél})
	\item az ezt létrehozó factory interfésze
	\item egy osztály (későbbiekben \emph{host}), ami implementálja a factory interfészt, illetve más szolgáltatásokat vesz igénybe; ő maga is egy szolgáltatás
	\item host privát, belső osztálya, ami implementálja célt a host által nyújtott szolgáltatások és kontextus felhasználásával.
\end{listing}







\section{Builder}

Amennyiben van egy komplex objektumunk, ami több komponenst is magában foglalhat, érdemes elválasztani az objektum létrehozását az objektum viselkedésének implementációjától \cite{gamma1995elements} . Amennyiben az objektum módot ad 

\begin{listing}
	\item egy egyszerú alapreprezentáció létrehozására
	\item ezen reprezentáció kibővítésére
\end{listing}

létre tudunk hozni egy entitást, ami felügyeli ezen folyamatot. Ez az entitás, a \emph{Builder}, definiál egy létrehozási folyamatot, mely folyamat több, eltérő reprezentációt is létre tud hozni, és az ezt egységesítő interfészt. \par

Egy példa a Java standard könyvtárából a \code{StringBuilder}. Segítségével képesek vagyunk egy stringet felépíteni kisebb szerkezeti egységek hozzáadásával, majd az így kapott karaktersorozatot megkapni. Az \code{insert} és \code{append} metódusai számos overloaddal rendelkeznek, így rugalmasabban tudjuk felépíteni a kívánt végeredményt. \par

Egy további példa a projectből a \code{NavBar} és a \code{NavBarBuilder}. A \code{NavBar} egy egységes navigációs sáv komponens, amelyet minden felhasználói csoport felülete használ. Egy opcionális címkéből, ami valamilyen információt ad a felületről, ahol megjelenik a sáv, illetve kattintható gombok sorozatából áll, melyek az alkalmazás egy megfelelő felületére navigálják a felhasználót. A \code{NavBar} egy komplex objektum. Ez nem viselkedésében tükröződik, hanem abban, hogy a megadható navigációs opciók számát nem tudjuk egyértelműen deifniálni, igény szerint változhat.\par

Egy lehetséges megoldás lenne, ha több konstruktort hoznánk létre az osztálynak, mindegyikben kezelve egy-egy lehetséges igényt: \par

\begin{listing}
	\item ne legyen címkéje, ne legyenek opciók
	\item legyen címkéje, de ne legyenek opciók
	\item ne legyen címkéje, legyen valamennyi opciója
	\item legyen címkéje, legyen valamennyi opciója
\end{listing} \par

Ezen esetek számát le tudjuk csökkenteni, ha a konstruktor \par

\begin{listing}
	\item a címkét \code{Optional<String>} típusúként kezeli, azaz függetlenül attól, hogy kell-e címke vagy sem, a konstruktor rendelkezni fog ezzel a paraméterrel
	\item az opciókat változó hosszúságú paraméterlistaként adjuk meg, amely, abban az esetben, ha nincs navigációs opció, üres, másképp pedig az opciókat tartalmazza.
\end{listing} \par

\snippet{NavBar}{A NavBar osztály implementációja}

Ezzel lecsökkentettük az objektum létrehozásának módjainak számát, azonban ez a megoldás nem rugalmas. Az objektum létrehozásához minden információra szükségünk van ami a címkét és az opciókat illeti, és lehetséges, hogy ezek nem állnak rendelkezésre egyszerre, vagy csak egy részük és ezek áthídalása jelentősen növeli a kódbázisunk komplexitását. \par

A \code{NavBarBuilder} bevezetésével a létrehozási folyamat jelentősen leegyszerűsödik. Definiálunk egy egységes interfészt, amely tartalmazza a \code{NavBar} létrehozásának lépéseit.

\snippet{NavBarBuilder}{NavBarBuilder}

Ezek a metódusok nem csak egyszerűbbé teszik a létrehozás folyamatát, hanem elrejtik a \code{NavBar} osztály valós implementációjának részleteit. A \code{NavBarBuilder} egy lehetséges implementációját a \ref{snippet:NavBarBuilderImpl} kódcsipetben láthatjuk.

\snippet{NavBarBuilderImpl}{A NavBarBuilder egy lehetséges implementációja}

\subsection{Adapter}

Az \emph{Adapter} minta felhasználásával egy osztály interfészét képesek vagyunk átalakítani valamilyen egyéb interfészre, amit a kliens definiált. Lehetővé teszi egyébként inkompatibilis interfészek és komponensek együttes használatát \cite{gamma1995elements}. \par

Az alkalmazás fejlesztése során felhasználtunk egy Vaadin add-ont, a Crud UI Add-ont, amely egy komponenst biztosít adatbázis műveletekhez szükséges felhasználói felületek létrehozására. Az add-on definiál egy interfészt \code{CrudListener<T>} néven, melynek egy implementációját át kell adnunk a már említett komponens konstruktorának.

\snippet{CrudListener}{A CrudListener interfész metódusai}

Ezek a metódusok könnyen térképezhetőek adatbázis műveletekre, nem növelné jelentősen a komplexitás mértékét az alkalmazásunk szellemi modelljében, ha erre az interfészre hagyatkozna az adatelérési, illetve szolgáltatás réteg, mint egységes interfész. Azonban ez a dizájn erős kapcsolatot hozna létre a már említett rétegek, illetve egy külső, UI könyvtár között, amelynek szerepét számos okból kifolyólag átveheti valamilyen más megoldás, más interfésszel, ami ismét nem lesz kompatibilis az általunk fejlesztett szoftverentitások interfészével. \par

\snippet{EntityService}{Egységes interfész adatbázis entitásokat kezelő szolgáltatásoknak}

Ahhoz, hogy a szolgáltatás réteg osztályainak állandó, más szoftveres komponensektől független interfészt tudjunk meghatározni, és mégis képesek legyünk ezen szolgáltalásokat használni a Crud UI-jal, be kell vezetnünk egy közvetítő osztályt a két fél közé. \par

Mivel a Java nem támogatja a több osztályból való öröklődést, ezért csupán az \code{Adapter} minta \emph{objektum adapter} variánsát tudjuk felhasználni. Ebben a verzióban létrehozunk egy osztályt, ami adapterként fog viselkedni a két interfész között, ami implementálja a célinterfészt (esetünkben a \code{CrudListener}-t), illetve rendelkezik egy mezővel, ami a kiinduló interfész egy példánya (esetünkben ez a \code{EntityService}). Ezen mezőt a konstruktor paramétereként kapott példánnyal inicializáljuk. Az adapter osztály a kiinduló interfész metódusait felhasználva implementálja a célinterfész metódusait. \par

\snippet{EntityServiceAdapter}{Az adapter osztály}

A későbbiekben amikor a célinterfész egy példányára van szükségünk, példányosítjuk az adapter osztályt, a kiinduló interfész valamely implementációjával.



\subsection{Strategy}

A \emph{Strategy} viselkedési minta lehetővé teszi, hogy definiálva egy általános algoritmus interfészét, algoritmusok egész családját hozhatjuk létre, melyek mindegyike egy lehetséges, érvényes implementációját enkapszulálja az algoritmusnak, és kölcsönösen felcserélhetővé teszi őket. A minta lehetővé teszi az algoritmus interfészének és valós implementációinak elválasztását, ami azt eredményezi, hogy ezek bármikor különbözhetnek kliensektől, amelyek felhasználják, anélkül, hogy a külvilág számára észlelhető viselkedésük inkonzisztens lenne \cite{gamma1995elements}. \par

\image{Általános példa a \emph{Strategy} mintára}{strategy.drawio.png}

Az alkalmazás fejlesztése során a publikus funkcióként elérhető keresés funkció implementálásában használtuk fel. Az általános algoritmus kívánt működése az, hogy termékek egy halmazából, előre meghatározott keresési szempontok alapján meghatározza azokat a termékeket, amelyek a keresési szempontok szerint megfelelnek bizonyos kritériumoknak. \par

Ezen algoritmus köré kiépíthetünk egy strategy mintát, ahol a mintában szereplő általános algoritmust a következő interfészben definiáljuk: 

\snippet{Strategy}{Egységes interface az algoritmus elérésére}

Láthatjuk, hogy egy egységes interface mellett, amelyen keresztül elérjük az algoritmus éppen aktuális implementációját, definiáljuk azt is, hogy milyen szempontok szerint történik a keresés. Alternatív megoldás lehetett volna, ha a \code{filterSearch} nem kéri ezeket paraméterként, hanem az egyes implementáló osztályok saját maguk definiálják az általuk vizsgált szempontokat, de ez a megoldás nem lenne rugalmas. \par

 Bár így lehetséges, hogy egy implementáció olyan keresési szempontot is kap, amely számára irreleváns, az egyes implementációk egymás között teljesen felcserélhetőek. \par
 
 Jelen formájában az alkalmazás csak egyfajta keresést támogat. Akkor fogadunk el egy terméket találatként, ha:
 
 \begin{listing}
 	\item \code{name} mezőjének részsztringje a \code{properties} \code{name} mezője
 	\item \code{priceInHuf} mezőjének értéke nagyobb vagy egyenlő a \code{properties} minPrice mezőjénél
 	\item \code{priceInHuf} mezőjének értéke kisebb vagy egyenlő a \code{properties} maxPrice mezőjénél
 \end{listing}

Ezen definíció mentén a következőképpen definiáltuk az algoritmus implementációját: 

\snippet{FilterStrategy}{A SearchStrategy egy lehetséges implementációja}

Mivel a keresési szempontok is egységesítésre kerültek, így a felhasználói felület tervezésekor hagyatkozhatunk ezekre a szempontokra, egységesen vagyunk képesek validálni őket. \par

Amennyiben másfajta keresési algoritmust szeretnénk használni, például egy olyan megoldást, amiben valamilyen \emph{fuzzy search}-ön \cite{hall1980approximate} alapuló algoritmussal vizsgáljuk, hogy az éppen vizsgált elem \code{name} mezője illeszkedik-e az erre vonatkozó keresési feltételre., ezt könnyedén megtehetjük.




\section{Használt könyvtárak és keretrendszerek}

A lombok könyvtár, aminek célja a repetatív kódrészletek (úgynevezett boilerplate kód) írásának elkerülése, a fejlesztői élmény javítása. A legtöbb esetben nem terjeszti ki a Java funkcióit, hanem már meglévő funkciók használatát teszi kényelmesebbé. \par

Bizonyos keretrendszerek disziplínái megkövetelik például a getter-ek és setter-ek, bizonyos contructor-ok definiálását. Ilyen esetben hasznlhatjuk a lombokot, ami a build folyamatunkba beépülve, valid Java bytecode-ot generál automatikusan azokban az osztályokban, ahol bevezettük az annotációit.\par

Amennyiben kíváncsiak vagyunk arra, hogy a lombok milyen transzformációkat hajt végre a kódunkon, vagy egyszerűen csak meg akarunk válni tőle, és eltávolítani a függőségink közül, a kódbázisunkat pedig megtisztítani a lombok annotációktól, abban az esetben erre is van lehetőség. \par

A delombok nevű eszköz előállítja azokat az osztályok forráskódját, amikben lombok annotációt használtunk, eltávolítva az annotációkat és helyükre a velük ekvivalens kód kerül, amelyet eddig a lombok generált.


\subsubsection{@NoArgsConstructor}

A \lstinline|@NoArgsConstructor|-t egy osztályra helyezve egy argomentumok nélküli konstruktort fog generálni. Amennyiben ez nem lehetséges, például egy \lstinline|final| mező miatt, a generálási folyamat egy fordítási hibát fog eredményezni \cite{lombokConstructorDocumentation}. Ez megkerülhető úgy, ha az annotáció \lstinline|force| paraméterének \lstinline|true| értéket adunk meg. Ezzel elérjük, hogy a \lstinline|final| mezők is inicializálva lesznek \lstinline|0|, \lstinline|false| vagy \lstinline|null| értékkel, azonban olyan mezők esetében, a \lstinline|@NonNull| annotációval végzünk null vizsgálatot, ez a vizsgálat nem fog legenerálódni, megtörténni. 

\lstinputlisting[language=Java]{code/NoArgsConstructorBefore.java}

\lstinputlisting[language=Java]{code/NoArgsConstructorAfter.java}

A project elkészítése során Spring Data JPA entitás osztályokat láttunk el \lstinline|@NoArgsConstructor| annotációval, mivel a JPA megköveteli egy ilyen konstruktor létezését.

\subsubsection{@AllArgsConstructor}

A \lstinline|@AllArgsConstructor| egy konstruktort generál egy osztálynak, mezőinként egy paraméterrel. Amennyiben egy mező el van látva a \lstinline|@NonNull| annotációval, a generált konstruktor egy null check-et fog végezni azon a mezőn.

\lstinputlisting[language=Java]{code/AllArgsConstructorBefore.java}

\lstinputlisting[language=Java]{code/AllArgsConstructorAfter.java}

\subsubsection{@Getter, @Setter}

\subsubsection{@ToString}

\subsubsection{@Builder}

\subsubsection{@Slf4j}






\subsection{Spring}

A Spring keretrendszer a Java, illetve Jakarta EE alkalmazásfejlesztés de facto sztenderdjévé vált 2002-es megjelenése óta. A framework egyik legfontosabb funkciója a \emph{dependency injection} modellje, ami segíti a gyors alkalmazásfejlesztést, továbbá átláthatóbb, tisztább kódot eredményez. A Srping modulárisan van felépítve, amelyek olyan szolgltatásokat nyújtanak, mint az adatelérés, tesztelés és web integráció. Fejlesztőként nem vagyunk kényszerítve, hogy a keretrendszer által kínált összes komponenst egyszerre használjuk. A moduláris modell lehetővé teszi, hogy csupán a szükséges elemeket tartalmazza projectünk, attól függően, hogy az éppen aktuálisan fejlesztett alkalmazás mit igényel \cite{buildingSpringRest}.\par

A Spring portfólióhoz számos más project tartozik, mint például a Spring Security, Spring Data vagy a Spring Boot, melyek mindegyike a Spring framework által nyújtott infrastruktúrára épül. Ezek célja rendre az autentikáció és autorizáció, az adatelérés és a Spring alkalmazások létrehozásának egyszerűbbé, elérhetőbbé tétele. \par

\subsubsection{Spring Security}

Az alkalmazásban az autentikációt és autorizációt Spring Security segítségével oldottuk meg.

\snippet{SecurityConfig}{Spring Security konfiguráció}

Láthatjuk, hogy van lehetőségünk Java kóddal is konfigurálni a Spring Security-t, a kódcsipetben látható módon. Képesek vagyunk URL minták megadásával meghatározni, hogy milyen felhasználói jogkör szükséges az adott erőforrás eléréséhez. Az alkalmazás könnyű bővíthetőségét segíti, ha a végpontok URL-jének meghatározásakor figyelembe vesszük, hogy milyen funkcióhoz és szerepkörhöz tartoznak, ezzel szemantikus jelentőséget adva az URL-eknek és egyúttal egy réteg alapú architektúrát hozunk létre. Ebben az esetben

\begin{listing}
	\item a Vaadin framework beéső működéséhez szükséges erőforrások autentikáció nélkül elérhető
	\item a gyökér URL, ahova az oldal felkeresése esetén először érkezünk publikusan elérhető
	\item bármely read-only API autentikáció nélkül elérhető
	\item az applikáció publikus funkciói autentikáció nélkül elérhetőek
	\item az applikáció éttermekhez tartozó funkciói eléréséhez \emph{RESTAURANT} szerepkörrel kell rendelkeznie a felhasználónak
	\item az applikáció végfelhasználókhoz tartozó funkciói eléréséhez \emph{END\_USER} szerepkörrel kell rendelkeznie a felhasználónak
\end{listing} \par

Amennyiben valamely szerepkör funkcióját szeretnénk bővíteni, például egy GraphQL alapú API-t létrehozni, vagy valamilyen új lehetőséggel augmentálni a az éttermeket, ezt könnyen és egszerűen megtehetjük ezen módszer bővíthetősége miatt. \par

\subsubsection{Spring Boot}

A Spring által biztosított dependency injection model az \code{@Autowired} annotáció segítségével érjük el. Ezzel a megoldással csak a Sprning által kezelt entitásokat tudjuk injektálni, azaz olyan osztályokat, amelyeknek van no-args-konstruktora, illetve el van látva valamilyen Spring által nyújtott \emph{stereotype annotációval}. Ezek kötül a legfontosabbak a \code{@Component}, \code{@Repository} és a \code{@Service}. Ez utóbbi kettő kiterjeszti a \code{@Component} szerepkörét. \par

Egy \code{@Component}-el annotált osztályokat automatikusan detektálja a Spring, mint általa kezelt komponens. A \code{@Repository} annyiban terjeszti ki ezt a viselkedési módot, hogy az adatelérési rétegből érkező, perzisztenciához kapcsolódó, platform specifikus hibákat elkapja és a Spring egységes kivételeként dobja tovább. A \code{@Service} jelzi a Spring számára, hogy az annotált osztály a szolgáltatás rétegbe tartozik, és üzleti logikát tartalmazhat, de jelenleg ennek nincsen szemantikus jelentősége. A Spring nem kényszeríti ránk ezen annotációk szerepkörüknek megfelelő idiomatikus használatát, azonban a jövőben érkezhetnek olyan változtatások a keretrendszerben, amely feltételezi az annotációk előírt használatát. \par

Az \code{@Autowired} annotációt osztályok mezőin, konstruktorokon, illetve setter metódusokon helyezhetjük el.

\snippet{AutowiredField}{Mezőn elhelyezett \code{@Autowired}}
\snippet{AutowiredSetter}{Setter-en elhelyezett \code{@Autowired}} 
\snippet{AutowiredConstructor}{Konstruktoron elhelyezett \code{@Autowired}}

A konstruktor alapú megoldás preferálandó, amennyiben szeretnénk, ha az injektált komponenshez tartozó mező \code{final} lenne, vagy ha csak az objektum létrejöttekor van szükségünk rá. \par

Az \code{@Autowired} típus alapján rezolválja az inkektálásra alkalmas osztályokat. Amennyiben több osztály is alkalmas erre, valamilyen módon jeleznünk kell a Spring felé, hogy melyik implementációt kívánjuk használni. \par

Ezt megtehetjük úgy, hogy a komponens deklarálásakor egy egyedi azonosítóval látjuk el, majd később erre hivatkozunk.

\snippet{Qualifier}{Azonosítóval ellátott komponens}

Egy alternatív megoldás ha az injektált mező neveként az injektálni kívánt implementáció nevét használjuk.

\snippet{BeanName}{Osztály neve, mint azonosító}


A Spring Boot Starter Data JPA egyszerűbbé teszi az adatbázisokkal való kommunikációt, egyszerűbbé teszi az adatelérési réteg kialakítását, mindezt oly módon, hogy a megoldás átlátható, könnyen búvíthető legyen. Amennyiben úgy döntünk, hogy nem kívánunk mi magunk adatbázis sémákat, entitásokat, lekéréseket létrehozni, hanem ezt a feladatot a Spring Data JPA-nak delegáljuk, lehetőségünk van arra, hogy a JPA hozza létre az adattáblákat, lekérdezéseket és egyéb SQL parancsokat, és ezek eredményét kezelje.

\subsubsection{Perzisztens entitások}

Ahhoz, hogy perzisztálhassuk egy osztály egy példányát egy adatbázisban 

\begin{listing}
	\item annotálnunk kell a \code{@Entity} annotációval
	\item rendelkeznie kell egy argomentumok nélküli konstruktorral
	\item minden perzisztálni kívánt mezőjének rendelkeznie kell publikus getter-el és setter-el
	\item egy mezőt meg kell jelölnünk az \code{@Id} annotációval, ami az adatbázisbeli azonosítója lesz
\end{listing}

Opcionálisan annotálhatjuk az entitás osztályt a \code{@Table} annotációval, amelynek \code{name} paramétereként megadott string lesz az entitásokat tartalmazó tábla neve. \par

Az entitás egy-egy mezőjéhez tartozó oszlop neve a mező nevével ekvivalens, amennyiben ezt nem írjuk felül a mezőn elhelyezett \code{@Column} annotáció \code{name} paraméterének adott string-gel. \par

Ügyeljünk arra, hogy semmiképen sem használjunk SQL kulcsszavakat táblák vagy mezők neveként.

\subsubsection{@Converter, AttributeConverter}

Amennyiben egy olyan értéket akarunk perzisztálni, ami nem

\begin{listing}
	\item JPA entitás
	\item JPA entitások kollekciója
	\item primitív Java típusok és ezek wrapper osztályai, illetve \code{String}
\end{listing}

\noindent biztosítanunk kell egy osztályt, ami definiál egy kölcsönös leképezést egy perzisztálható és a jelenlegi perzisztálni kívánt típus között. \par

Ennek eléréséhez definiálnunk kell egy osztályt, ami annotálva van a \code{@Converter} annotációval, illetve implementálja a \code{AttributeConverter<A, C>} generikus interfészt, ahol A az az entitás attribútum típus, amiből kiindulunk, és perzisztálni kívánjuk, C pedig az, amit valóban el tud tárolni az adatbázis. \par

Az \code{AttributeConverter<S, T>} interfésznek két definiált metódusa van:

\begin{listing}
	\item \code{C convertToDatabaseColumn(A attribute)}, ez konvertálja az entitásból kapott attribútumot egy, adatbázis által tárolható értékké
	\item \code{A convertToEntityAttribute(C column)}, ez konvertálja az adatbázis egy oszlopában szereplő értéket entitás attribútummá.
\end{listing}

Az interfészt implementáló osztályt annotálnunk kell a \code{@Converter} annotációval annak érdekében, hogy alkalmazható legyen az átváltás. Ammenyiben az annotáció \code{autoApply} paraméterének igaz értéket adunk, a perzisztancia ellátójának (angolul \emph{persistence provider}) muszály automatikusan alkalmazni a konvertert minden entitás minden perzisztált attribútumára, amire alkalmazható, kivéve, ahol az attribútumon elhelyezett \code{@Convert} annotáció ezt felülírja \cite{converterDocumentation}.

A project elkészítése során \code{enum} értékek konvertálására használtuk.

\snippet{DishType}{DishType}
\snippet{DishTypeConverter}{DishTypeConverter}

\subsubsection{@OneToOne}

\subsubsection{@OneToMany, @ManyToOne}

\subsubsection{@ManyToMany}

\subsubsection{JpaRepository, @Repository}

A Spring Data JPA által biztosított \code{JpaRepository} egyszerű, könnyen használható és kibővíthető megoldást nyújt az adatelérési réteg létrehozására. Csupán egy interface-t kell létrehoznunk, ami kiterjeszti a \code{JpaRepository<T, ID>} generikus interfészt, ahol a T a perzisztált entitás típusa, ID pedig ezen entitás azonosító mezőjének típusa \cite{jpaRepositoryDocumentation}. A \code{JpaRepository} által definiált metódusok lehetővé teszik az alapvető  DUCS adatbázis műveletek (azaz Delete, Update, Create, Select) használatát perzisztált entitásokra. \par

Amennyiben az alap műveleteken túlmutató lekérdezéseket szeretnénk definiálni, erre is van módunk. A \code{JpaRepository}-t kiterjesztő interfészben van módunk további metódusokat deklarálni, amelyek, ha nevük bizonyos szemantikai szabályokat követnek, interpretálva lesznek, mint SQL query-k. Az említett szemantikai szabályok az \ref{tab:JpaRepository} táblázatban találhatóak. \par

A \code{@Repository} annotációval jelezzük a Spring keretrendszer felé, hogy az interfész egy \emph{Repository}, azaz egy mechanizmus entitások tárolásra, kinyerésésre és keresésésre. A Spring 2.5-ös verzója óta a \code{@Component} annotáció egy specializációjaként is szolgál, azaz imlpementációi automatikusan detektálódnak.



\subsection{Vaadin}

A Vaadin keretrendszerek egy családja, ami a Flow illetve a Hilla (korábban Fusion) front-end keretrendszereket foglalja magában. Az általunk használt Flow framework lehetővé teszi reszponzív, interaktív felhasználói felületek létrehozását kizárólag Javában, komponensek segítségével. Számos beépített komponenssel és funkcióval rendelkezik, illetve számos közösség által fejlesztett plug-inja van. \par

A beépített komponensek között találhatóak HTML-ből ismerős elemek, mint például az

\begin{listing}
	\item \code{Input}
	\item \code{TextArea}
	\item \code{Button}
	\item \code{Div}
	\item \code{H1, H2}
	\item \code{Hr}
\end{listing}

Emellett biztosít olyan komponenseket, amelyeket nem tudunk megfeleltetni HTML elemeknek, azonban hasonló konstrukciókkal már találkozhattunk.

\begin{listing}
	\item \code{HorizontalLayout}, egy flex container, aminek \code{flex-direction} tulajdonsága \code{row} értékű 
	\item \code{VerticalLayout}, egy flex container, aminek \code{flex-direction} tulajdonsága \code{column} értékű 
	\item \code{PasswordField}, az \code{<input>} elem egy specializációja
	\item \code{NumberField}, az \code{<input>} elem egy specializációja
\end{listing}

A Vaadin lehetőséget nyújt továbbá \emph{route}-ok definiálására és az ezek közötti navigációra.

\snippet{Route}{a \code{app/restaurant/staff} route-hoz tartozó nézet definíciója}

Ezek között a \code{UI} osztály \code{navigate} metódusával tudunk navigálni, úgy, hogy a metódusnak átadjuk paraméterként a cél route-hoz tartozó nézetet definiáló osztály referenciáját.

\snippet{Navigate}{Gombnyomásra visszanavigálunk a rendelésekhez}

\section{Felhasznált szabványok, ajánlások}

\subsection{UUID}

Az univerzálisan egyedi azonosító (angolul \emph{Universally unique identifier}) vagy UUID egy ITF szabvány, amit az RFC 4122 definiált. A fő motiváció használatára az, hogy nem szükséges egy központi szerv bevonása a létrejövő azonosítók adminisztrálására, így generálásuk teljes mértékben automatizálható. A UUID-k fix mérete miatt, ami 128 bit, jelentősen kisebb, mint a legtöb alternatív megoldás. Kompakt méretéből következik, hogy használata optimálisabb teljesítményt eredményez rendező és hasító algoritmusok, illetve adatbázisban való tárolás esetében. Az említett RFC-ben leírt generáló algoritmus akár másodpercenként 10 milló allokációt tud elvégezni gépenként, melyből kifolyólag a UUID-t tranzakciók azonosítójaként is használható. \par

A UUID azonosítók valójában nem \emph{teljesen} egyediek, azaz van esély arra, hogy két generált azonosító ugyanazzal az értékkel fog rendelkezni, ütközni fognak. Azonban ennek a valószínűsége elenyésző. A UUID ütközést tekinthetjük a születésnap probléma egy speciális esetének. Annak az esélye, hogy egy populációban, egymástól függetlenül kiosztott $x$ azonosító közül $n$-et kiválasztva $p$ valószínűséggel legyen köztük legalább kettő egyező az

\begin{equation*}
	n = 0.5 + \sqrt{0.25 - 2\times(\ln q)\times x}
\end{equation*}

formulával kiválóan lehet közelíteni \cite{mathis1991generalized}, ahol $q = 1 - p$. Ebből adódóan ahhoz, hogy legalább 50\%-os eséllyel generálódjon legalább két UUID,

\begin{equation*}
	n \approx 0.5 + \sqrt{0.25 + 2\times(\ln 2)\times2^{122}} \approx 2.71 \times 10^{18}
\end{equation*}

azonosítót kellene generálnunk. Ehhez, a másodpercenkénti 10 milló generált azonosítóból kiindulva megközelítóleg 8587 év szükséges, azaz ennyi idő suükséges ahhoz, hogy 50\%-os valószínűséggel előidézzükn egy UUID ütközést.

\subsection{REST API}

Az állapotreprezentáció-transzfer (angolul \emph{representational state transfer}) vagy REST nem egy konkrét, jól definiált standard, sokkal inkább egy architektúrális stílus, ami elterjedt technológiákat és szabványokat használ fel web alapú szolgáltatások tervezésére és implementálására \cite{richards2006representational}. \par

\emph{Roy Thomas Fielding} doktori disszertációban számos megkötést tett arra, hogy hogyan definiálható egy REST szolgáltatás architektúrája \cite{fielding2000architectural}. 

\subsubsection{Kliens-szerver architektúra}
Emögött a \emph{separation-of-concerns} elv áll. Azzal, hogy elválasztjuk a felhasználói interfészt az adattárolás szerepkörétől, skálázhatóbbá és szélesebb körben portolhatóvá a felhasználói interfész. Emellett ez az elkülönülés megengedi, hogy külön a szerverkomponensek egymástól elkülönülve fejlődjenek.

\subsubsection{Állapotmentes}
A kliens és szerver közti kommunikációnak állapotmentesnek kell lennie, oly módon, hogy a klienstől érkező minden kérése tartalmazza az összes, a kérés feldolgozásához szükséges információt, nem hagyatkozhat bármilyen, a szerveren tárolt kontextusra. \par

Ez a döntés növelte a 
\begin{listing}
	\item láthatóságot, egy megfigyelő (vagy \emph{monitoring}) rendszernek csupán egy kérés alapján meg tudja határozni a kérés teljes természetét
	\item megbízhatóságot, mivel a rendszer könnyebben helyreáll egy részleges hiba után
	\item skálázhatóságot, mivel azzal, hogy nem tárol a szerver kérések között semmilyen állapotot, gyorsabban fel tudja szabadítani használt erőforrásait, illetve nem kell menedzselnie az erőforrásokat akár több kérésen keresztül
\end{listing}

Ezzekkel szemben negatívumként említhetjük a romlott hálózati teljesítményt.

\subsubsection{Gyorsítótár}

Az előbb említett hátrányra válaszul vezessük be a modellünkbe gyprsítótárazhatóság fogalmát. Ez megköveteli hogy egy válaszban szereplő adat implicit vagy explicit módon meg legyen jelölve hogy gyorsítótárazható-e. Ha egy válasz gyorsítótárazható, akkor egy kliens oldali gyorsítótárnak módjában áll újra felhasználni azt az adatot későbbi, az eredeti kéréssel ekvivalens kérésekhez.

\subsubsection{Egységes interfész}

Azzal, hogy a REST szolgáltatások egy egységes interfészen kommunikálnka a klienssel, függetlenítjuk a klienset a szolgáltatás implementációjától. Annak érdekében, hogy egy interenet szintű REST szolgáltatásoknak egy egységes interfészt határozzunk meg, (azaz egy egyezményt kliens és a szolgáltatás között, ami definiálja kommunkációjuk formáját), ezt szabványok felhasználásával kell megtennünk.

\begin{listing}
	\item Erőforrás azonosítás: URI standard \cite{RFC3986}
	\item Erőforrás manipuláció: HTTP standard \cite{RFC2616}
	\item Önleíró üzenetek: MIME típusok \cite{RFC2045}
	\item HATEOAS: hyperlink-ek és URI template-ek \cite{RFC6570}
\end{listing}

\subsubsection{Réteg alapú rendszer}

Hierarchikus rétegeket alakítunk ki azáltal, hogy az adott rétegek csak a velük közvetlenül kapcsolatban lévő rétegekről tudnak, csak ezekkel tudnak kommunikálni. Azzal, hogy limitáljuk az egyes rétegek tudását a rendszerről, csökken az egész rendszer komplexitása. A rétegek alapú architektúra adta lehetőségeket használhatjuk legacy szolgáltatások, komponensek szeparálására, illetve megkönnyíti az ezekről való átállást.

\subsubsection{REST és HTTP}

A REST szolgáltatások HTTP kérések fogadása és válaszok küldése útján bonyolítják le a kliens és szerver közti kommunikációt. A HTTP kérések és válaszok szemantikai jelentését az \ref{tab:restAndHttp} ábrán láthatjuk.

\section{BCrypt}

A 90-es évek végén a mikroprocesszorok gyorsulása rohamosan növelte kibertámadások mögött álló számítási teljesítményt, míg számos autentikációs séma titkos, felhasználók által válaszott jelszavakon alapult, amelyek hossza és véletlenszerűsége közel konstans maradt. Niels Provos és David Mazières erre a problémára megoldásként egy jövőbiztos, a hardveres fejlődéssel lépést tartó jelszó sémát, és ennek részeként a BCrypt algoritmust hozta létre. \cite{provos1999future}. \par

Az algoritmust szándékosan lassúra és költségesre tervezték. Ez jó designbeli döntésnek bizonyult, mivel ezzel csökkenthető a \emph{brute-force}, azaz nyers számításierőn alapuló támadások hatékonysága. A BCrypt által nyújtott védelem tovább növelhető azzal, ha az algoritmust többször lefuttatjuk, minden új körben az előzőből kapott értéket véve hash-elendő értéknek. Ezzel a technikával még tovább inkrementálható a titkosítás hatékonysága. \par

A projektben először SHA-256 algoritmussal titkosítottuk a felhasználók jelszavát. Az emögött álló érvek a gyors, hatékony működés, illetve az alacsony erőforrásigény voltak. A későbbiekben viszont pont ezekből az okokból kifolyólag, illetve mivel hatékonysága jelentős teljesítménybeli növekedést élvez, ha GPU-n implementáljuk. Ez a tulajdonsága jelentősen növeli a brute-force alapú támadások hatékonyságát \cite{patra2021cryptography}. A BCrypt algoritmus nem rendelkezik ezzel a tulajdonsággal, nehéz hatékonyan implementálni GPU-n. Azzal, hogy egy eleve lassabb, nem párhuzamosítható algoritmust használunk, amit könnyen tudunk skálázni a több körös titkosítással, nem rontjuk sem az alkalmazást rendeltetésszerűen használó felhasználók élményét, sem az alkalmazás hatékonyságát. Amennyiben egy jogosult személy kívánja magát autentikálni, az algoritmus futási ideje szinte elhanyagolható lesz. \par

\subsection{Web Push API}

Az alkalmazásban szeretnénk, hogyha az étterem kap valamilyen értesítést arról, amint új rendelés érkezik be, a kliens pedig, ehhez hasonlóan, kapjon valamilyen vizuálisan megjelenő visszajelzést a felhasználói felületen, ha az étterem feldolgozta a rendelést, vagy a rendelés elkészült; amennyiben pedig a felhasználók fizetni szeretnének, ezen szándékukat tudják jelezni oly módon, hogy az étterem azonnal értesüljön erről, és a tranzakció lebonyolításához szükséges minden információ rendelkezésükre álljon. \par

Ezen probléma megoldására valamilyen alkalmazáson belül eseménykezelő rendszert kell létrehoznunk, ami alkalmas előre meghatározott üzenettípusok küldésére, ezekre való feliratkozásra (azaz bejövő üzenetek figyelésére), illetve ezen üzenetek kezelésére. Létrehoztunk egy sugárzó-feliratkozó (angolul \emph{boradcaster-subscriber}) alapú event kezelő rendszert, ahol a feliratkozók egy callback megadásával képesek feliratkozni eventekre, a sugárzók pedig a kód bármely pontjából képesek eventeket sugározni, az event értelmezéséhez szükséges minden információval együtt. \par

Mivel ezek az eventek a szerveren jönnek létre, de hatásukat a klienseknél, azaz a felhasználói felületen fejtik ki, ezért megoldást kell találnunk arra, hogyan teremtsünk gyors kommunikációt a kettő között. Egy lehetséges megoldás a kliens oldali poll-ozás lenne. A kliens előre meghatároztt időközönként kérést indít a szerver felé valamilyen információért, például hogy feldolgozásra került-e egy rendelés, vagy elkészült-e, és a szervertől kapott válasz alapján cselekszik.

A probléma ezzel a megoldással, hogy nem hatékony. Amennyiben túl hosszú a kérés újraküldésének ideje, a kliens unreszponzívnak tűnhet, lassan reagál a szerver oldali változásokra. Amennyiben pedig ezen időintervallum túl rövid, a szerverünknek túl sok kérést kell kezelnie, és ezen kérések egy része "feleslges", hiszen nem történt állapotváltozás az előző kérés feldolgozása óta. Tpvábbá függetlenül attól, hogy milyen gyakran poll-ozunk, ez a folyamat jelentős hálózati forgalmat fog generálni és amennyiben a kapcsolat nem megfelelő erősségű, ez jelentősen ronthatja a felhasználói élményt és az alkalmazás hatékonyságát. \par

A megoldás abban rejlik, hogy a szerveren létrejövő változásokról (például egy rendelés állapota megváltozott) értesítést küldünk a klienseknek, akik ezt az értesítést fogadják és kezelik. Ezen rendszer létrehozásában a Vaadin által szolgáltatott \emph{Push} funkcióra hagyatkoztunk, ami a \emph{Push API}-on alapszik. \par
	
	\chapter{Összefoglaló}

	A dolgozat során lefejlesztettünk egy webalkalmazást, ami kihasználja a modern webtechnológiák nyújtotta lehetőségeket. Törekedtünk letisztult, zajmentes felhasználói interfészt létrehozni, úgy, hogy ez ne menjen a vizuális élmény, interaktivitás és használhatóság rovására. \par

Az, hogy az alkalmazást nem asztali alkalmazásként, hanem webes platformra tervezett megoldásként hoztuk létre helyes döntésnek bizonyult. A központi backend lehetővé tette olyan funkciók megalkotását, melyek nem jöhettek volna létre egy lokálisan vagy decentralizáltan futó backend esetében. Mivel a kliens oldalon csak a felhasználói felület megjelenítése történik, csupán egy alkalmas megjelenítő eszköz, illetve egy böngésző szükséges az alkalmazás használatához. Elkerültük a telepítés nehézkes folyamatait, továbbá a frissítések anélkül érkeznek meg a felhasználókhoz, hogy nekik ezzel kapcsolatban bármilyen teendőjük lenne. \par

A Java, illetve a JVM, mint szoftveres platform jó választásnak bizonyult. A fejlesztői közösség aktív és segítőkész, a köréjük kiépült ökoszisztéma számos könyvtárat és megoldást biztosított problémáinkra. Az általunk használt megoldások mindegyike szabad és nyílt forráskódú szoftver, amiket aktívan karbantart a közösség. Jól dokumentált, kiforrott keretrendszerekről és könyvtárakról van szó, melyek mindegyike jelentős ideje elérhető. A karbantartók rengeteg időt fordítottak stabilitásuk garantálására, és funkcióik bővítésére. Számos könyv segítette használatukat, ezzel növelve a fejlesztés gyorsaságát. \par

A Spring keretrendszer kifejezetten hasznosnak bizonyult. Ez az ipari erősségű megoldás kiváló skálázható alkalmazások fejlesztésére. Nemcsak segíti, hanem jutalmazza is egy átgondolt, tiszta alkalmazásarchitektúra létrehozását, mindezt anélkül, hogy büntetné a fejlesztőt azért, ha mégis el szeretne térni az ajánlott idiómáktól. A Spring Security, illetve a Spring Boot Starter Data JPA modulok által biztosított funkciók lehetővé tették egy biztonságos, megbízható működésű, kiterjeszthető alkalmazás létrehozását. \par

A Vaadin szuboptimális választásnak bizonyult. Bár a mögötte álló fejlesztői csapat jelentős hangsúlyt fektet az innovációra, a keretrendszer nem rendelkezik kellő közösséggel. A keretrendszer weboldalán található fórumok inaktívak, jellemzőek a több éves, megoldatlan problémákról szóló szálak. A dokumentáció, bár nagyon hasznos azoknak, akik meg szeretnének ismerkedni a keretrendszerrel, nem tartalmaz komplexebb példákat. Sok esetben nem tér ki a sarkalatos esetekre, melyeknek feltárása egy hosszadalmas folyamatnak bizonyult. \par

A fejlesztési folyamatot magunk mögött tudva az alternatív frontend megoldások használata kecsegtetőbb. A legtöbb JavaScript frontend framework, legyen az akár React, Angular, VueJS vagy a Svelte, aktívabb közösséggel, jobb dokumentációval rendelkezik, számos könyv íródott alkalmazásukról. Bár használatukhoz rendelkeznünk kell HTML, CSS és JavaScript ismeretekkel az adott framework idiómái megértése mellett, ezek az alapok felhasználhatóak amikor esetleg áttérünk egyik megoldásról egy másikra. Továbbá az említett technológiák szabadságot nyújtanak ötleteink megvalósítására, mindezt rugalmas módon. \par

A fejlesztés során nagy szerepet játszottak a fejlesztői eszközök. Az Apache Maven, mint build eszköz, ipari szinten elterjedt, szabad és nyílt forrású megoldás, ami nagyban segítette a project létrejöttét, bővülését és tesztelését. Kiválóan integrálható Spring-el, Lombok-al, illetve a Vaadin keretrendszerrel. A JetBrains IntelliJ kényelmes és hasznos fejlesztői környezetnek bizonyult. Kiváló Maven, Lombok, Spring és git integrációja, továbbá a refaktorálás folyamatát elősegítő funkcióinak hála lehetőségünk volt a fejlesztésre koncentrálni, az egyes eszközök közti kontextusváltás elkerülésével. \par

Az alkalmazás részeként létrehozott read-only REST API egységes, kibővíthető interfészt biztosít arra, hogy más rendszerek is tudjanak kommunikálni az alkalmazással. Mivel a REST egy elterjedt architektúrális megoldás, számos cikk, könyv és szoftveres komponens segítette kialakítását. Bár definíciója nem egzakt, irányelveit követve valóban skálázható és megbízható megoldásokat tudtunk fejleszteni. \par

A lefejlesztett alkalmazás kiváló precedenst nyújtott modern webes technológiák, szoftverfejlesztési eszközök és dizájn minták kipróbálására és alkalmazására a gyakorlatban. A dolgozat célját ebből a szempontból elérte. Ami a létrejött alkalmazást illeti, minden kitűzött funkciót sikerüt implementálnunk, méghozzá oly módon, hogy a végeredmény biztonságos, megbízható és bővíthető legyen.
	
	\printbibliography[
	heading=bibnumbered, title={Irodalomjegyzék}]
	
	\chapter{Függelék}
	
	
	\footnotesize
	\begin{longtable}{| p{.20\linewidth} | p{.30\linewidth} | p{.50\linewidth } |}		
			\caption{JpaRepository kiterjesztéséhez  használható kulcsszavak}
			\label{tab:JpaRepository} \\
			\hline
			Kulcsszó & Minta & JPQL kódcsipet \\ 
			\hline
			And &	findByLastnameAndFirstname	& … where x.lastname = ?1 and x.firstname = ?2 \\
			\hline
			Or &	findByLastnameOrFirstname	& … where x.lastname = ?1 or x.firstname = ?2 \\
			\hline
			Is,Equals &	findByFirstname \linebreak findByFirstnameIs \linebreak findByFirstnameEquals	& … where x.firstname = 1? \\
			\hline
			Between &	findByStartDateBetween	& … where x.startDate between 1? and ?2 \\
			\hline
			LessThan &	findByAgeLessThan	& … where x.age < ?1 \\
			\hline
			LessThanEqual &	findByAgeLessThanEqual	& … where x.age <= ?1 \\
			\hline
			GreaterThan &	findByAgeGreaterThan	& … where x.age > ?1 \\
			\hline
			GreaterThanEqual &	findByAgeGreaterThanEqual	& … where x.age >= ?1 \\
			\hline
			After &	findByStartDateAfter	& … where x.startDate > ?1 \\
			\hline
			Before &	findByStartDateBefore	& … where x.startDate < ?1 \\
			\hline
			IsNull &	findByAgeIsNull	& … where x.age is null \\
			\hline
			IsNotNull,NotNull &	findByAge(Is)NotNull	& … where x.age not null \\
			\hline
			Like &	findByFirstnameLike	& … where x.firstname like ?1 \\
			\hline
			NotLike &	findByFirstnameNotLike	& … where x.firstname not like ?1 \\
			\hline
			StartingWith &	findByFirstnameStartingWith	& … where x.firstname like ?1 (parameter bound with appended \%) \\
			\hline
			EndingWith &	findByFirstnameEndingWith	& … where x.firstname like ?1 (parameter bound with prepended \%) \\
			\hline
			Containing &	findByFirstnameContaining	& … where x.firstname like ?1 (parameter bound wrapped in \%) \\
			\hline
			OrderBy &	findByAgeOrderByLastnameDesc	& … where x.age = ?1 order by x.lastname desc \\
			\hline
			Not &	findByLastnameNot	& … where x.lastname <> ?1 \\
			\hline
			In &	findByAgeIn(Collection<Age> \linebreak ages)	& … where x.age in ?1 \\
			\hline
			NotIn &	findByAgeNotIn(Collection<Age> \linebreak age)	& … where x.age not in ?1 \\
			\hline
			True &	findByActiveTrue()	& … where x.active = true \\
			\hline
			False &	findByActiveFalse()	& … where x.active = false \\
			\hline
			IgnoreCase &	findByFirstnameIgnoreCase	& … where UPPER(x.firstame) = UPPER(?1) \\
			\hline
	\end{longtable}
	
\end{document}